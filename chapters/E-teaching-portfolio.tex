\chapter{Teaching Qualifications Portfolio} \label{chap:teaching-qualifications-portfolio}

\section{Summary of teaching activities} \label{sec:summary-of-teaching-activities}

When I first came to Caltech in 2008, I researched the history of Richard Feynman's career at the school, and the one philosophy of his that resonated with me is \enquote{Lernen durch Lehren}, learning by teaching. Feynman once said, \enquote{If you want to master something, teach it. A great way to learn is to teach.} and I take it to heart. But honestly, I have always loved to teach. I take great pleasure in learning new concepts and explaining them to others, which helps reinforce what I learned. In a way, this is similar to a positive feedback loop and allows me to be self-critical, go back, and make sure I made no faulty assumptions or took anything for granted. On top of it all, I have had a relatively unique experience in academia (see my diversity statement!). My learned experience growing up as a deaf kid in oral schools gave me a mental checklist of guidelines for teaching others: be supportive, be humble, be kind, be accessible, be respectful, and be fun. I always feel like I can give back by being able to teach. The rest of this document will highlight how these tenets shape my role as an instructor: teaching through the American Red Cross, how I approached teaching classes at Caltech, some strategies I have employed as a Teaching Assistant at the University of Chicago, and my efforts to improve the software literacy of our members of the ATLAS Collaboration.

\section{Personal reflection on teaching} \label{sec:personal-reflection-on-teaching}

\subsection{Teaching Saves Lives} \label{ssec:teaching-saves-lives}
For eight years through to the end of my undergraduate studies at Caltech, I was an instructor and an instructor trainer with the American Red Cross (ARC) in the Palm Beach County (Florida) and San Gabriel Pomona Valley (California) chapters. I taught HIV/AIDS Peer Education, First Aid, and CPR to folks from all walks of life, cultures, and communities, and I also trained new instructors for the ARC. I found this incredibly rewarding since it allowed me to develop my public speaking skills and continually adapt my teaching style on the fly to fit the classroom. One notable example is when I traveled to Riveria Beach, FL, a predominantly poor neighborhood with little access to health prevention, to teach Infant/Child CPR/AED for new/young parents. I only had a few dozen previous classes under my belt when I ran into a roadblock. Usually, we teach the students to do compressions following the BeeGee song \textsl{Stayin' Alive} (admittedly a little tongue-in-cheek) because it has a 100 beats-per-minute rate. However, none of the parents knew that song, so we went through a list of artists they love, and someone mentioned they loved \textsl{Bohemian Rhapsody}; I asked if they knew Queen. They did, so I used \textsl{Another One Bites the Dust} as the song for them to follow along (yes, also tongue-in-cheek). I find that students retain the knowledge better when the class is stimulating and enjoyable for them.

\subsection{Teaching My Peers} \label{ssec:teaching-my-peers}
At Caltech, I had the opportunity through Dean John F. Hall to teach a class on Basic Web Programming in 2010 and 2011 to my fellow peers! This class allowed me to reinforce the up-and-coming technologies in web programming (at the time, this was very bleeding edge) and try out different teaching styles that I thought would suit my personality. The class involved 3 hours of work each week, with 1.5 hours for hands-on lecture (using a \textsl{flipped classroom} approach) and 1.5 hours of take-home work. I used the feedback from the class in 2010 to update and revise the course plan for the 2011 iteration. Professor K. Mani Chandy sponsored the course in 2010, Professor Adam Wierman in 2011. The main goal with this class was to ensure that I wasn't just teaching programming but that real-world examples informed the concepts. I find it challenging to learn new ideas using unrealistic models or abstractions, so I wanted to teach the class the best way I thought students would understand. If something did not work, I took constructive criticism to develop a better course next time.

\subsection{Approachable Education} \label{ssec:approachable-education}
As a graduate student at The University of Chicago, I led recitation sections for core physics and laboratory classes as a Teaching Assistant. Two former TAs at Caltech inspired my teaching style: Professors Harvey Newman and Steven Frautschi. Harvey would look for esoteric or tricky teasers and use that time to talk through them with a blackboard and reveal the way his brain approaches problems like these in physics. His goal was to impart critical thinking skills and recognize patterns. Steven encouraged students to come up and work through the physics problems in real-time and provided feedback on the steps in our solution. At UChicago, I led interactive sessions that had a mix of challenging and straightforward problems. I solicited real-time feedback and continually adjusted my lesson plan for that hour based on student engagement. I also imagined myself as a student in my classroom and thought about what resources I could provide as an instructor that would allow me, as a student, to be more engaged. I have made it a habit to always furnish a write-up/copy of my lecture notes to all students beforehand. I am also happy to provide some examples of these materials to the committee upon request.

\subsection{A Rising Tide Lifts All Boats} \label{ssec:a-rising-tide-lifts-all-boats}
In the ATLAS experiment, a key ingredient to being a productive particle physicist is to have enough technical expertise to perform data analysis. In an international, scientific collaboration, sometimes that expertise or knowledge slips through the cracks. Sometimes the student who worked on the software for an analysis graduates out! Luckily there are modern tools and industry-wide systems at our disposal to ensure that continuity happens, but this requires teaching our colleagues new tricks. %~\cite{Malik2021,irishep2020,gitlabStory2018}.
Since I started working in particle physics, I have instructed students, postdocs, junior and senior faculty at all skill levels from all socioeconomic backgrounds. I have organized and hosted boot camps to spend time on various software skills and techniques necessary to address the computing challenges in High Energy Physics (HEP). Participants come out of these workshops equipped to pursue or further their own research goals. They also come out with transferable skills to careers both inside and outside of HEP.

\subsection{My Teaching Philosophy} \label{ssec:my-teaching-philosophy}
I want to summarize my approach if I am a professor at Harvard. An essential aspect of my teaching philosophy is to recognize that all of us are involved in cutting-edge research, but we do not often share that passion in our classes. I will try to connect current research to the course material, although relating Hooke's law to non-perturbative views of hadronization in Quantum Chromodynamics is a stretch. Where possible, I will seek out various real-world examples to demonstrate the physics concepts I would be teaching to let my passion for physics shine through. Hands-on activities keep the students engaged and keep the class fun. I will support and expand physics outreach efforts in the local community, especially underrepresented populations, using my experience teaching particle physics to local schools via ATLAS Masterclass. Finally, I aim to keep my lectures approachable and accessible because I am acutely aware that my classrooms will be full of potential colleagues and students from all walks of life. I am satisfied when I can provide students with the same opportunities and access to education that I have been fortunate to receive.

\section{List of teaching qualifications} \label{sec:list-of-teaching-qualifications}

\subsection{Formal training in teaching and learning in higher education \none}\label{ssec:formal-training-in-teaching-and-learning-in-higher-education-none}
\subsection{Educational training relevant to the subject, or other training in teaching and learning \none}\label{ssec:educational-training-relevant-to-the-subject-or-other-training-in-teaching-and-learning-none}
\subsection{Other experience of an educational nature \none}\label{ssec:other-experience-of-an-educational-nature-none}
% which the applicant considers wholly or partially relevant or complementary to the other points (point of view to be justified)
\subsection{Teaching experience or equivalent}\label{ssec:teaching-experience-or-equivalent}
\subsection{Supervision at the Bachelor’s and Master’s degree levels}\label{ssec:supervision-at-the-bachelor-s-and-master-s-degree-levels}

During my postdoc tenure at UC Santa Cruz and my studies at UChicago, I mentored undergraduate students across various analysis and hardware projects. I list graduate students that I mentored in~\Cref{sssec:experience-as-an-assistant-supervisor}. Please note that I have not had the opportunity to form my own research group, so I have no experience being a formal principal supervisor of undergraduate students.

\begin{table}[h!]
	\footnotesize
	\centering
	\caption{List of undergraduate students that I mentored during my tenure, along with a brief description of what the research topic(s) were.}
	\begin{tabular}{l|>{\bfseries}r|>{\itshape}l|p{20em}}
		Sam Kelson         & 2024         & IRIS-HEP Fellow & coffea schemas for ATLAS data format: \texttt{PHYSLITE}                                                   \\
		Zach Pizzo         & 2024         & UC Santa Cruz   & ITk Pixels Upgrade: Type-1 services cable assembly                                                        \\
		Samantha Contreras & 2023-2024    & UC Santa Cruz   & ITk Pixels Upgrade: visual inspection of pixel modules                                                    \\
		Marco Frank        & 2023-present & UC Santa Cruz   & ITk Pixels Upgrade: electrical testing of pixel modules                                                   \\
		Scott Philips      & 2023-present & UC Santa Cruz   & ITk Pixels Upgrade: electrical testing of pixel modules                                                   \\
		Sambridhi Deo      & 2023         & IRIS-HEP Fellow & REANA: implement workflow for galaxy rotation-curve fitting analysis                                      \\
		Keaton Ferguson    & 2022-2024    & UC Santa Cruz   & ITk Pixels Upgrade: technician for pixel modules and pixel services testing                               \\
		Bo Zheng           & 2020         & IRIS-HEP Fellow & Hardware acceleration of statistical fitting with GPUs/TPUs                                               \\
		Noah Peake         & 2019-2022    & UC Santa Cruz   & ITk Pixels Upgrade: technician for pixel modules and pixel services testing                               \\
		Henry Zheng        & 2018         & UChicago        & gFEX: System-on-Chip Development (custom OS and firmware)                                                 \\
		Ben Warren         & 2018         & UChicago        & gFEX: development of Machine Learning algorithms to run on GPU                                            \\
		Brandon Nadal      & 2017         & UChicago        & gFEX: REU MRSEC fellow, development of software to monitor system statistics and environmental conditions \\
		Natalie Harrison   & 2015-2017    & UChicago        & Supersymmetry: design of recursive jigsaw selections for strong production of SUSY                        \\
		Daniel Sullivan    & 2015-2016    & UChicago        & gFEX: implementing firmware for histogramming and monitoring of real-time data processing                 \\
	\end{tabular}
\end{table}

\subsection{Educational leadership \noneyet}\label{ssec:educational-leadership-noneyet}
\subsection{Educational development work}\label{ssec:educational-development-work}

I have developed a lot of education material as can be seen in~\Cref{ssec:symposia-conferences-workshops-and-collaborations}. In addition to these presentations, I have created some material or full-fledged courses to train the next generation of physicists. As a postdoc at UC Santa Cruz, I was not afforded many opportunities to teach in a formal setting; so I sought out opportunities to do so through my experimental collaborations.

\Activity{pyhf tutorial}{2018-present}{pyhf}{GitHub: \href{https://pyhf.github.io/pyhf-tutorial/introduction.html}{\faIcon{external-link-alt}}}
\begin{itemize}
	\item Wrote pedagogical material for how to use \texttt{pyhf} from a mathematical perspective
	\item Continuously developing new material based on mentor interactions with my students to bridge gaps in statistical knowledge for high energy physics
\end{itemize}

\Activity{GitLab CI/CD}{2018-present}{HEP Software Foundation and Software Carpentries}{GitHub: \href{https://hsf-training.github.io/hsf-training-cicd/}{\faIcon{external-link-alt}}}
\begin{itemize}
	\item Wrote a brand new lesson on how to use GitLab Continuous Integration / Continuous Deployment in particle physics
	\item Continuously updating the online course based on student feedback whenever I teach the class
	\item Inspired an equivalent GitHub CI/CD version, with the same structure, but translated to be compatible with GitHub Actions
\end{itemize}

\Activity{Web Programming}{2010-2011}{California Institute of Technology}{Pasadena, CA (US)}
\begin{itemize}
	\item One of ten total student taught classes that have been done at Caltech
	\item Designed the syllabus and all training materials for the course (including presentations, assignments, and final exams)
	\item Graded all student work, and provided final evaluations for credit
	\item Taught this course to bridge a gap in Caltech's limited offerings in computer science
\end{itemize}

\Activity{Computational Physics Lab}{Mar 2011 - Jun 2011}{California Institute of Technology}{Pasadena, CA (US)}
\begin{itemize}
	\item Migrated the core software \enquote{CurveFit} for the computational physics lab from an older version of Mathematica to a newer version of Mathematica
	\item Developed a rubric for TAs to evaluate lab students in a consistent, equitable manner
\end{itemize}

\Activity{Information Systems and Technology}{Mar 2010 - Jun 2011}{California Institute of Technology}{Pasadena, CA (US)}
\begin{itemize}
	\item Lead teaching assistant for Prof. Shuki Bruck for one of Caltech's most renowned elective courses
	\item Provided two-hour Office Hour sessions weekly to assist with homework, answer questions about lectures, and improve students' understanding
	\item Structured and graded homework assignments for approximately 140 students
\end{itemize}

\subsection{Production of teaching materials and publications}\label{ssec:production-of-teaching-materials-and-publications}

\Activity{Graduate Student Teaching Assistant}{2012-2017}{University of Chicago}{Chicago, IL (US)}
\begin{itemize}
	\item TAed for six different courses: Advanced Physics Laboratory (Fall 2016-2017), Advanced Electromagnetism (Winter 2014-2015), Advanced Mechanics (Fall 2013-2014), Mechanics (Summer 2012-2013), Special Relativity and Electromagnetism (Winter 2012-2013), and Introductory Mechanics (Fall 2012-2013)
	\item Discussion sections occurred weekly, and I prepared a set of \enquote{lecture} notes and distributed it ahead of each session
	\item Developed and graded weekly lab reports, problem sets, midterms, and final exams.
	\item Converted handwritten lecture notes to \LaTeX for \enquote{Advanced Electromagnetism}
\end{itemize}

\Activity{Bridge Program}{2014-2016}{University of Chicago}{Chicago, IL (US)}
\begin{itemize}
	\item Mentored underrepresented youth from local high school students in the Chicago Public system
	\item Developed individualized strategies to match each student's needs and create a pathway for their entry to college
\end{itemize}

\subsection{National and international educational work \noneyet}\label{ssec:national-and-international-educational-work-noneyet}

One of my core ideals is to make physics more accessible to all. One successful way I find is through teaching others. I have been involved in organizing and instructing at many workshops, bootcamps, tutorials as both a graduate student and as a postdoc. I approach teaching by trying to teach others the way I wish I was taught the concepts, such that it is both intuitive and memorable. \Cref{ssec:symposia-conferences-workshops-and-collaborations} has a list highlighting the varying amount of instruction that I have done since 2016.

\subsection{Internationalisation work within teaching practice \noneyet}\label{ssec:internationalisation-work-within-teaching-practice-noneyet}
\subsection{Reporting assignments and evaluation assignments \noneyet}\label{ssec:reporting-assignments-and-evaluation-assignments-noneyet}
\subsection{Symposia, conferences, workshops and collaborations}\label{ssec:symposia-conferences-workshops-and-collaborations}

\begin{enumerate}[label=\PubCite{red!50!black}{white}{\arabic*},resume]
	\item \Presentation{CAMPFIRE}{How to do an ATLAS Analysis}{https://indico.cern.ch/event/1360774/}{Jun 2024}
	\item \Presentation{ATLAS ITk Week}{Tutorial: Access to PDB backup at CERN}{https://indico.cern.ch/event/1395925/}{Apr 2024}
	\item \Presentation{pyhf Users and Developers Workshop}{organizer and instructor}{https://indico.cern.ch/event/1294577/}{Dec 2023}
	\item \Presentation{US-ATLAS Workshop @ SLAC}{Systematic Uncertainties}{https://indico.cern.ch/event/1309794/}{Oct 2023}
	\item \Presentation{PyHEP}{pyhf tutorial and exploration}{https://indico.cern.ch/event/1252095/}{Oct 2023}
	\item \Presentation{Reinterpretation Forum}{Reduce, Reuse, Reinterpret (mapyde)}{https://conference.ippp.dur.ac.uk/event/1178/}{Aug 2023}
	\item \Presentation{CAMPFIRE}{How to do an ATLAS Analysis}{https://indico.cern.ch/event/1258921/}{Jul 2023}
	\item \Presentation{Reinterpretation Forum}{MaPyDe + ATLAS SimpleAnalysis}{https://indico.cern.ch/event/1197680/}{Dec 2022}
	\item \Presentation{ATLAS Exotics Workshop}{Data (Products) Preservation}{https://indico.cern.ch/event/1147662/}{Sep 2022}
	\item \Presentation{DANCE/CoDaS @ Snowmass 2022}{instructor}{https://indico.cern.ch/event/1151329/}{Jul 2022}
	\item \Presentation{ATLAS ITk Week}{Using the Production Database API}{https://indico.cern.ch/event/1149581/}{May 2022}
	\item \Presentation{US-ATLAS Computing Bootcamp}{organizer and instructor}{https://indico.cern.ch/event/1075740/}{Oct 2021}
	\item \Presentation{PyHEP}{Distributed statistical inference with pyhf}{https://indico.cern.ch/event/1019958/}{Jul 2021}
	\item \Presentation{ATLAS Induction Day and Software Tutorial}{Intro to pyhf and hands-on}{https://indico.cern.ch/event/1042811/}{Jul 2021}
	\item \Presentation{ATLAS Induction Day and Software Tutorial}{Using GitLab for Analysis}{https://indico.cern.ch/event/1042811/}{Jul 2021}
	\item \Presentation{SUSY+HDBS+Exotics RECAST Tutorial}{organizer and instructor}{https://indico.cern.ch/event/1009271/}{Mar 2021}
	\item \Presentation{HSF / IRIS-HEP GitHub CI/CD Training}{instructor}{https://indico.cern.ch/event/1001128/}{Feb 2021}
	\item \Presentation{Future Analysis Systems and Facilities}{Making an Analysis Pipeline}{https://indico.cern.ch/event/960587/}{Oct 2020}
	\item \Presentation{CMS B2G Workshop}{pyhf: tutorial and exploration}{https://indico.cern.ch/event/898965/}{Sep 2020}
	\item \Presentation{US-ATLAS Computing Bootcamp}{organizer and instructor}{https://indico.cern.ch/event/933434/}{Aug 2020}
	\item \Presentation{ATLAS Canada Computing Workshop}{GitLab CI and Statistical Analysis}{https://indico.cern.ch/event/892952/}{Jul 2020}
	\item \Presentation{HSF / IRIS-HEP Virtual Pipelines Training}{instructor}{https://indico.cern.ch/event/904759/}{Jun 2020}
	\item \Presentation{Analysis Preservation Bootcamp}{One of the core instructors for the analysis preservation workshop at CERN}{https://indico.cern.ch/e/awesome}{Feb 2020}
	\item \Presentation{ATLAS Induction Day and Software Tutorial}{Intro to pyhf and hands-on}{https://indico.cern.ch/event/860971/}{Jan 2020}
	\item \Presentation{ATLAS Induction Day and Software Tutorial}{Using GitLab for Analysis}{https://indico.cern.ch/event/860971/}{Jan 2020}
	\item \Presentation{ATLAS Induction Day and Software Tutorial}{Intro to pyhf and hands-on}{https://indico.cern.ch/event/831761/}{Oct 2019}
	\item \Presentation{ATLAS Induction Day and Software Tutorial}{Using GitLab for Analysis}{https://indico.cern.ch/event/831761/}{Oct 2019}
	\item \Presentation{USATLAS/FIRST-HEP Computing Bootcamp}{organizer and instructor}{https://indico.cern.ch/event/816946/}{Aug 2019}
	\item \Presentation{Analysis Systems Topical Workshop}{Likelihood publishing \& Reinterpretation}{https://indico.cern.ch/event/822074/}{Jun 2019}
	\item \Presentation{ATLAS Induction Day and Software Tutorial}{Using GitLab for Analysis}{https://indico.cern.ch/event/772589/}{Jan 2019}
	\item \Presentation{US-ATLAS Hadronic Final State Forum}{Jet and MET triggers}{https://indico.cern.ch/event/747355/}{Dec 2018}
	\item \Presentation{ATLAS Machine Learning Workshop}{Intro to pyhf and hands-on}{https://indico.cern.ch/event/735932/}{Oct 2018}
	\item \Presentation{ATLAS Induction Day and Software Tutorial}{Using GitLab for Analysis}{https://indico.cern.ch/event/757797/}{Oct 2018}
	\item \Presentation{ATLAS Induction Day and Software Tutorial}{Using GitLab for Analysis}{https://indico.cern.ch/event/731951/}{Jul 2018}
	\item \Presentation{ATLAS Exotics Workshop}{Tips and tricks for GitLab CI}{https://indico.cern.ch/event/709718/}{May 2018}
	\item \Presentation{ATLAS Software Tutorial}{Using GitLab for Analysis}{https://indico.cern.ch/event/710710/}{Apr 2018}
	\item \Presentation{US ATLAS Induction Week}{Using GitLab for Analysis Code and Continuous Integration}{https://indico.cern.ch/event/684668}{Jan 2018}
	\item \Presentation{ATLAS S\&C Documentation Workshop}{Modern doc tools and approaches}{https://indico.cern.ch/event/649804/}{Dec 2017}
	\item \Presentation{ATLAS P\&P Week with 120ifb: ASG Parallel Session}{Using GitLab for Analysis Code Management}{https://indico.cern.ch/event/681128/}{Dec 2017}
	\item \Presentation{UChicago EFI Data Analytics workshop}{Python, Jupyter, and ROOT}{https://indico.cern.ch/event/676637/}{Oct 2017}
	\item \Presentation{ATLAS Hadronic Calibration Workshop}{Analysis Optimization}{https://indico.cern.ch/event/642438/}{Aug 2017}
	\item \Presentation{ATLAS Software Tutorial}{multiple tutorials}{https://indico.cern.ch/event/524296/}{Jun 2016}
	\item \Presentation{UChicago/ASC-ANL software tutorial}{multiple tutorials}{https://indico.cern.ch/event/463456/}{Mar 2016}
\end{enumerate}

\subsection{Distinctions and awards for educational activities}\label{ssec:distinctions-and-awards-for-educational-activities}

Below are a list of awards and distinctions associated with my teaching activity.

\Award{UC Santa Cruz Outstanding Postdoctoral Fellow Award}{May 2022}{\enquote{The two postdoctoral scholars chosen by the selection committee to receive the award will exhibit the following:
		\begin{itemize}
			\setlength{\itemsep}{0em}
			\item Excellent research and/or creative activity, showing strong evidence of research/creative innovation and productivity (e.g., scholarly distinctions, publications, presentations, inventions, exhibits, performances, products), as well as the nominee’s research/creative innovation and productivity having a significant or potentially significant impact on the field and/or society more broadly
			\item Leadership and/or strong service (e.g., preparing manuscripts, funding applications, organizing workshops or conferences, volunteering in professional and/or other organizations, etc.)
			\item Effective mentorship by advising graduate students, undergraduate students, and/or any other group in a professional setting
			\item Support and fostering of equity, diversity, and inclusion within their research group, department, institution, community, and/or field through their research/creativity, mentoring/teaching, and/or service activities
		\end{itemize}} [Citation from UC Santa Cruz]}
\Award{US ATLAS Outstanding Graduate Student Award}{Jun 2016}{\enquote{In recognition of your exceptionally broad and noteworthy contributions to the ATLAS experiment. In particular, we recognize your critical contributions to the electronics design and prototyping for a new high-speed trigger electronics system for the Phase 1 upgrade, software development, leadership in the creation of a new method to search for Supersymmetry, and software education.} [Citation from US ATLAS]}
\Award{Caltech Excellent TA Award}{Jun 2012}{\enquote{Giordon is an UG in Physics and clearly knew of his students appreciation for his help as his section grew and grew in attendance. \textsl{The G man rocks!} was expressed, as students appreciated his clear explanations and enthusiasm. A student stated, \textsl{He is not only engaging, but the material he presents in class is easy enough to understand and challenging enough to reinforce our understanding of the topic.}} [Citation from Caltech]}
\Award{Palm Beach Post -- Best of Class 2008}{Apr 2008}{\enquote{Giordon's other passion is his volunteer work with the American Red Cross. He teaches elementary-school kids about first aid and rescue breathing, and his enthusiasm earned him one of three youth member positions on the Palm Beach County Chapter's board out of more than 300 applicants.} [Citation from Palm Beach Post]}
\Award{Howard Shavel Youth Award}{Apr 2006}{Award presented by the American Red Cross chapter of the Greater Palm Beach County area for recognition of exceptional volunteerism. I was recognized for my efforts in both teaching CPR/First Aid/Sexual Education classes and training new instructors to teach those classes \& lifeguard training.}

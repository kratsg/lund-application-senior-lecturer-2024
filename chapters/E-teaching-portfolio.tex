\chapter{Teaching Qualifications Portfolio} \label{chap:teaching-qualifications-portfolio}

\section{Summary of teaching activities} \label{sec:summary-of-teaching-activities}

When I first came to Caltech in 2008, I researched the history of Richard Feynman's career at the school, and the one philosophy of his that resonated with me is \enquote{Lernen durch Lehren}, learning by teaching. Feynman once said, \enquote{If you want to master something, teach it. A great way to learn is to teach.} and I take it to heart. But honestly, I have always loved to teach. I take great pleasure in learning new concepts and explaining them to others, which helps reinforce what I learned. In a way, this is similar to a positive feedback loop and allows me to be self-critical, go back, and make sure I made no faulty assumptions or took anything for granted. On top of it all, I have had a relatively unique experience in academia (see my diversity statement!). My learned experience growing up as a deaf kid in oral schools gave me a mental checklist of guidelines for teaching others: be supportive, be humble, be kind, be accessible, be respectful, and be fun. I always feel like I can give back by being able to teach. The rest of this document will highlight how these tenets shape my role as an instructor: teaching through the American Red Cross, how I approached teaching classes at Caltech, some strategies I have employed as a Teaching Assistant at the University of Chicago, and my efforts to improve the software literacy of our members of the ATLAS Collaboration.

\section{Personal reflection on teaching} \label{sec:personal-reflection-on-teaching}

\subsection{Teaching Saves Lives} \label{ssec:teaching-saves-lives}
For eight years through to the end of my undergraduate studies at Caltech, I was an instructor and an instructor trainer with the American Red Cross (ARC) in the Palm Beach County (Florida) and San Gabriel Pomona Valley (California) chapters. I taught HIV/AIDS Peer Education, First Aid, and CPR to folks from all walks of life, cultures, and communities, and I also trained new instructors for the ARC. I found this incredibly rewarding since it allowed me to develop my public speaking skills and continually adapt my teaching style on the fly to fit the classroom. One notable example is when I traveled to Riveria Beach, FL, a predominantly poor neighborhood with little access to health prevention, to teach Infant/Child CPR/AED for new/young parents. I only had a few dozen previous classes under my belt when I ran into a roadblock. Usually, we teach the students to do compressions following the BeeGee song \textsl{Stayin' Alive} (admittedly a little tongue-in-cheek) because it has a 100 beats-per-minute rate. However, none of the parents knew that song, so we went through a list of artists they love, and someone mentioned they loved \textsl{Bohemian Rhapsody}; I asked if they knew Queen. They did, so I used \textsl{Another One Bites the Dust} as the song for them to follow along (yes, also tongue-in-cheek). I find that students retain the knowledge better when the class is stimulating and enjoyable for them.

\subsection{Teaching My Peers} \label{ssec:teaching-my-peers}
At Caltech, I had the opportunity through Dean John F. Hall to teach a class on Basic Web Programming in 2010 and 2011 to my fellow peers! This class allowed me to reinforce the up-and-coming technologies in web programming (at the time, this was very bleeding edge) and try out different teaching styles that I thought would suit my personality. The class involved 3 hours of work each week, with 1.5 hours for hands-on lecture (using a \textsl{flipped classroom} approach) and 1.5 hours of take-home work. I used the feedback from the class in 2010 to update and revise the course plan for the 2011 iteration. Professor K. Mani Chandy sponsored the course in 2010, Professor Adam Wierman in 2011. The main goal with this class was to ensure that I wasn't just teaching programming but that real-world examples informed the concepts. I find it challenging to learn new ideas using unrealistic models or abstractions, so I wanted to teach the class the best way I thought students would understand. If something did not work, I took constructive criticism to develop a better course next time.

\subsection{Approachable Education} \label{ssec:approachable-education}
As a graduate student at The University of Chicago, I led recitation sections for core physics and laboratory classes as a Teaching Assistant. Two former TAs at Caltech inspired my teaching style: Professors Harvey Newman and Steven Frautschi. Harvey would look for esoteric or tricky teasers and use that time to talk through them with a blackboard and reveal the way his brain approaches problems like these in physics. His goal was to impart critical thinking skills and recognize patterns. Steven encouraged students to come up and work through the physics problems in real-time and provided feedback on the steps in our solution. At UChicago, I led interactive sessions that had a mix of challenging and straightforward problems. I solicited real-time feedback and continually adjusted my lesson plan for that hour based on student engagement. I also imagined myself as a student in my classroom and thought about what resources I could provide as an instructor that would allow me, as a student, to be more engaged. I have made it a habit to always furnish a write-up/copy of my lecture notes to all students beforehand. I am also happy to provide some examples of these materials to the committee upon request.

\subsection{A Rising Tide Lifts All Boats} \label{ssec:a-rising-tide-lifts-all-boats}
In the ATLAS experiment, a key ingredient to being a productive particle physicist is to have enough technical expertise to perform data analysis. In an international, scientific collaboration, sometimes that expertise or knowledge slips through the cracks. Sometimes the student who worked on the software for an analysis graduates out! Luckily there are modern tools and industry-wide systems at our disposal to ensure that continuity happens, but this requires teaching our colleagues new tricks. %~\cite{Malik2021,irishep2020,gitlabStory2018}.
Since I started working in particle physics, I have instructed students, postdocs, junior and senior faculty at all skill levels from all socioeconomic backgrounds. I have organized and hosted boot camps to spend time on various software skills and techniques necessary to address the computing challenges in High Energy Physics (HEP). Participants come out of these workshops equipped to pursue or further their own research goals. They also come out with transferable skills to careers both inside and outside of HEP.

\subsection{My Teaching Philosophy} \label{ssec:my-teaching-philosophy}
I want to summarize my approach if I am a professor at Harvard. An essential aspect of my teaching philosophy is to recognize that all of us are involved in cutting-edge research, but we do not often share that passion in our classes. I will try to connect current research to the course material, although relating Hooke's law to non-perturbative views of hadronization in Quantum Chromodynamics is a stretch. Where possible, I will seek out various real-world examples to demonstrate the physics concepts I would be teaching to let my passion for physics shine through. Hands-on activities keep the students engaged and keep the class fun. I will support and expand physics outreach efforts in the local community, especially underrepresented populations, using my experience teaching particle physics to local schools via ATLAS Masterclass. Finally, I aim to keep my lectures approachable and accessible because I am acutely aware that my classrooms will be full of potential colleagues and students from all walks of life. I am satisfied when I can provide students with the same opportunities and access to education that I have been fortunate to receive.

\section{List of teaching qualifications} \label{sec:list-of-teaching-qualifications}

\subsection{Formal training in teaching and learning in higher education \none}\label{ssec:formal-training-in-teaching-and-learning-in-higher-education-none}
\subsection{Educational training relevant to the subject, or other training in teaching and learning \none}\label{ssec:educational-training-relevant-to-the-subject-or-other-training-in-teaching-and-learning-none}
\subsection{Other experience of an educational nature \none}\label{ssec:other-experience-of-an-educational-nature-none}
% which the applicant considers wholly or partially relevant or complementary to the other points (point of view to be justified)
\subsection{Teaching experience or equivalent}\label{ssec:teaching-experience-or-equivalent}
\subsection{Supervision at the Bachelor’s and Master’s degree levels}\label{ssec:supervision-at-the-bachelor-s-and-master-s-degree-levels}

During my postdoc tenure at UC Santa Cruz and my studies at UChicago, I mentored undergraduate students across various analysis and hardware projects. I list graduate students that I mentored in~\Cref{sssec:experience-as-an-assistant-supervisor}. Please note that I have not had the opportunity to form my own research group, so I have no experience being a formal principal supervisor of undergraduate students.

\begin{table}[h!]
	\footnotesize
	\centering
	\caption{List of undergraduate students that I mentored during my tenure, along with a brief description of what the research topic(s) were.}
	\begin{tabular}{l|>{\bfseries}r|>{\itshape}l|p{20em}}
		Sam Kelson         & 2024         & IRIS-HEP Fellow & coffea schemas for ATLAS data format: \texttt{PHYSLITE}                                                   \\
		Zach Pizzo         & 2024         & UC Santa Cruz   & ITk Pixels Upgrade: Type-1 services cable assembly                                                        \\
		Samantha Contreras & 2023-2024    & UC Santa Cruz   & ITk Pixels Upgrade: visual inspection of pixel modules                                                    \\
		Marco Frank        & 2023-present & UC Santa Cruz   & ITk Pixels Upgrade: electrical testing of pixel modules                                                   \\
		Scott Philips      & 2023-present & UC Santa Cruz   & ITk Pixels Upgrade: electrical testing of pixel modules                                                   \\
		Sambridhi Deo      & 2023         & IRIS-HEP Fellow & REANA: implement workflow for galaxy rotation-curve fitting analysis                                      \\
		Keaton Ferguson    & 2022-2024    & UC Santa Cruz   & ITk Pixels Upgrade: technician for pixel modules and pixel services testing                               \\
		Bo Zheng           & 2020         & IRIS-HEP Fellow & Hardware acceleration of statistical fitting with GPUs/TPUs                                               \\
		Noah Peake         & 2019-2022    & UC Santa Cruz   & ITk Pixels Upgrade: technician for pixel modules and pixel services testing                               \\
		Henry Zheng        & 2018         & UChicago        & gFEX: System-on-Chip Development (custom OS and firmware)                                                 \\
		Ben Warren         & 2018         & UChicago        & gFEX: development of Machine Learning algorithms to run on GPU                                            \\
		Brandon Nadal      & 2017         & UChicago        & gFEX: REU MRSEC fellow, development of software to monitor system statistics and environmental conditions \\
		Natalie Harrison   & 2015-2017    & UChicago        & Supersymmetry: design of recursive jigsaw selections for strong production of SUSY                        \\
		Daniel Sullivan    & 2015-2016    & UChicago        & gFEX: implementing firmware for histogramming and monitoring of real-time data processing                 \\
	\end{tabular}
\end{table}

\subsection{Educational leadership}\label{ssec:educational-leadership}
\subsection{Educational development work}\label{ssec:educational-development-work}
\subsection{Production of teaching materials and publications}\label{ssec:production-of-teaching-materials-and-publications}
\subsection{National and international educational work}\label{ssec:national-and-international-educational-work}
\subsection{Internationalisation work within teaching practice}\label{ssec:internationalisation-work-within-teaching-practice}
\subsection{Reporting assignments and evaluation assignments}\label{ssec:reporting-assignments-and-evaluation-assignments}
\subsection{Symposia, conferences, workshops and collaborations}\label{ssec:symposia-conferences-workshops-and-collaborations}

\begin{enumerate}[label=\PubCite{green!50!black}{white}{\arabic*},resume]
	\item \Presentation{Analysis Preservation Bootcamp}{One of the core instructors for the analysis preservation workshop at CERN}{https://indico.cern.ch/e/awesome}{February, 2020}
	\item \Presentation{USATLAS/FIRST-HEP Computing Bootcamp}{One of the core instructors for the USATLAS Software Bootcamp at LBNL}{https://indico.cern.ch/event/816946/}{August, 2019}
	\item \Presentation{Using GitLab for Analysis Code and Continuous Integration}{US ATLAS Induction Week, early 2018}{https://indico.cern.ch/event/684668}{January, 2018}
	\item \Presentation{Using GitLab for Analysis Code Management}{ATLAS P\&P Week with 120ifb: ASG Parallel Session}{https://indico.cern.ch/event/681128/}{December, 2017}
\end{enumerate}

% add in other training stuff

\subsection{Distinctions and awards for educational activities}\label{ssec:distinctions-and-awards-for-educational-activities}

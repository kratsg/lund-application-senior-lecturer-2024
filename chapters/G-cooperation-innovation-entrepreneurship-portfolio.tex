\chapter{Portfolio of Cooperation with Wider Society, Innovation, and Entrepreneurship}

\section{Summary of cooperation, innovation, and entrepreneurship} \label{sec:summary-of-cooperation-innovation-and-entrepreneurship}

I will be upfront and state that this is likely my weakest area, and I want to continue increasing my engagement and visibility outside of academia. My cooperation with wider society is broken up into a few broad areas: popular media in technology and Deaf awareness including memes, open data and analysis preservation, lobbying with legislators, and being a role model for Deaf in the sciences. I create and share my own particle physics inspired memes and work with groups like ASL Core to continue developing more signs for physics. I regularly engage with the Deaf community by hosting weekly coffee chats and pizza chats, as well as monthly \enquote{Deaf Night Out} at local bars in the area. However, while these engagements are not strictly tied to engaging the public in particle physics, I use them to continue meeting new people and honing my ability to explain my research in an approachable way, and answer any questions they might have given most of the Deaf community is aware of my profile as a particle physicist. Finally, I was involved in a trip to the Federal Government to meet legislative representatives in order to lobby for funding. I enjoyed this and would love more opportunities to be able to advocate for others in my field in this way.

\section{Cooperation, innovation, and entrepreneurship -– personal reflection} \label{sec:cooperation-innovation-and-entrepreneurship-personal-reflection}

Elevating scientific literacy within wider society stands as a cornerstone for societal progress and prosperity. By fostering a deeper understanding of scientific principles, methodologies, and discoveries among the general populace, we empower individuals to make informed decisions about critical issues ranging from healthcare to environmental sustainability. Moreover, collaboration with wider society, including organizations like the Swedish National Association of the Deaf (SDR), fosters inclusivity and accessibility in educational outreach efforts. In Sweden, renowned for its culture of innovation and entrepreneurship, this collaboration is particularly vital. By engaging with diverse stakeholders, including industry partners, policymakers, and community organizations, academia can harness collective creativity and expertise to drive innovation and address pressing societal challenges. This synergy not only enriches research endeavors but also fuels economic growth and societal development, positioning Sweden as a global leader in scientific innovation. As a prospective faculty member at Lund University, I am deeply committed to promoting scientific literacy, fostering inclusivity through collaboration with organizations like SDR, and leveraging educational outreach to inspire future generations and catalyze innovation for the betterment of Sweden and beyond.

\section{Cooperation, innovation, and entrepreneurship – list of qualifications} \label{sec:cooperation-innovation-and-entrepreneurship-list-of-qualifications}

\subsection{Formal training in media and communication \none} \label{ssec:formal-training-in-media-and-communication-none}
\subsection{Information to business/culture sector/associations/industry/public sector \noneyet} \label{ssec:information-to-business-culture-sector-associations-industry-public-sector-noneyet}
\subsection{Advice to business/culture sector/associations/industry/public sector} \label{ssec:advice-to-business-culture-sector-associations-industry-public-sector}

\Activity{Spacing out with Particle Physics: Deaf Space Camp}{Apr 2024}{U.S. Space \& Rocket Center}{Huntsville, AL (US)}
\begin{itemize}
	\setlength{\itemsep}{0em}
	\item Collaborated with the U.S. Space and Rocket Center, NASA, and CSD Learns
	\item Facilitated Space Camp Unlimited for DHH middle and high school students
	\item Four DHH role models, each possessing expertise in space and STEM-related areas, are chosen for this years' camp
	\item Created an introductory video, showcasing my background, experience, expertise, and advice/tips, filmed remotely through the OpenReel platform
	\item Participated in a live panel discussion during the Space Camp Unlimited program, providing insightful answers to questions and interacting with students.
	\item Introductory videos are to be showcased to Space Camp Unlimited students and the public via CSD Learns' social media platforms and outreach efforts.
\end{itemize}

\Activity{Developing a Lexicon for Physics}{2012-present}{ASLCore, RIT}{Rochester, NY (US)}
\begin{itemize}
	\setlength{\itemsep}{0em}
	\item Contributed to the development of ASLCore curriculum at Rochester Institute of Technology (RIT), focusing on expanding the lexicon to encompass physics-related terminology
	\item Collaborated with a team of linguists, sign language interpreters, and Deaf content experts (e.g., myself) to integrate physics-specific signs into the ASLCore lexicon, ensuring linguistic accuracy and cultural relevance
	\item Participated in workshops aimed at refining and enhancing the ASLCore curriculum, providing valuable insights and feedback to optimize the learning experience for students.
	\item Engaged in the creation and validation of new signs for physics concepts, fostering effective communication and comprehension within the ASL-using community.
	\item Played an instrumental role in promoting ASL proficiency by incorporating physics-related signs into the ASLCore curriculum, empowering students to express complex scientific concepts fluently.
\end{itemize}

\Activity{Federal Government: Lobbying for Particle Physics}{Mar 2015}{U.S. LHC Users Association, University Research Associates}{Washington D.C. (US)}
\begin{itemize}
	\setlength{\itemsep}{0em}
	\item Annual trip where dozens of physicists descend upon the Federal Government to visit House of Representatives and the Senate and discuss importance of funding Particle Physics in the US
	\item Vast majority of US HEP research is supported by taxpayer funding
	\item Congress appropriates funding for our programs with the expectation that our basic research will pay off now (via side benefits) and in the future (via enabling completely new science and technology)
	\item We communicate with Congress directly for several reasons:
	      \begin{itemize}
		      \item Provide information about our program’s impacts, status, successes, plans
		      \item Convey our gratitude for their and taxpayers’ support
		      \item Reinforce community support for our budget ask
	      \end{itemize}
	\item Coordinated by Fermilab Users Executive Committee, US LHC Users Association, SLAC Users Organization, and APS Division of Particles and Fields
	\item Fermilab handles communication training for this, with a few hours of presentations and practice talks to get used to speaking with politicians and legislative staff
\end{itemize}

\subsection{Development of information and educational material for the general public, other professional groups, etc. \noneyet} \label{ssec:development-of-information-and-educational-material-for-the-general-public-other-professional-groups-etc-noneyet}
\subsection{Participation in various media} \label{ssec:participation-in-various-media}

\begin{itemize}
	\setlength{\itemsep}{0.25em}
	\item \Presentation{Spacing out with Particle Physics}{CSD Learns}{https://atomichands.com/events/space-camp-unlimited/}{Apr 2024}
	\item \Presentation{The last great mystery of the mind: meet the people who have unusual – or non-existent – inner voices}{The Guardian - Sirin Kale}{https://web.archive.org/web/20211025125540/https://www.theguardian.com/science/2021/oct/25/the-last-great-mystery-of-the-mind-meet-the-people-who-have-unusual-or-non-existent-inner-voices}{Oct 2021}
	\item \Presentation{PARTY CALL PHYSICS: when access and physics collide}{A Keynote Speaker for the International Conference of Physics Students, 2021}{https://events.iaps.info/event/9/page/5-keynote-speakers}{Aug 2021}
	\item \Presentation{Deaf scientists thrive with interpreters and technology}{Physics Today - Adria Schwarber}{https://web.archive.org/web/20210723175238/https://physicstoday.scitation.org/do/10.1063/PT.6.4.20210723a/full/}{Jul 2021}
	\item \Presentation{PSD climate grants foster belonging while socially distanced}{The University of Chicago - Physical Sciences Division}{https://web.archive.org/web/20210415190429/https://physicalsciences.uchicago.edu/news/article/psd-climate-grants-foster-belonging-while-socially-distanced/}{Apr 2021}
	\item \Presentation{ATLAS releases \enquote{full orchestra} of analysis instruments}{Symmetry Magazine}{https://web.archive.org/web/20210115133429/https://www.symmetrymagazine.org/article/atlas-releases-full-orchestra-of-analysis-instruments}{Jan 2021}
	\item \Presentation{My Journey Obtaining a Physics PhD While Deaf}{Rochester Institute of Technology, World of Wonder in Science seminar series}{https://www.rit.edu/ntid/deafscientists/wow}{Apr 2020}
	\item \Presentation{My life as a particle physicist (in American Sign Language)}{CERN YouTube Channel, part of the former CERN Microcosm Exhibit}{https://youtu.be/3sESUT1UO6E}{Feb 2020}
	\item \Presentation{New open release streamlines interactions with theoretical physicists}{ATLAS Experiment}{https://web.archive.org/web/20201221150149/https://atlas.cern/updates/news/new-open-likelihoods}{Dec 2019}
	\item \Presentation{A matter of interpretation}{Symmetry Magazine}{http://web.archive.org/web/20191205000925/https://www.symmetrymagazine.org/article/a-matter-of-interpretation-asl-physics}{Dec 2019}
	\item \Presentation{Alternate-Universe ASEE: An Engineering Education Conference Session from a World where the Majority of Engineers are Deaf}{CDEI Distinguished Lecture Series, Annual American Society for Engineering Education}{https://diversity.asee.org/deicommittee/wp-content/uploads/sites/2/2020/03/CDEI-Newsletter_Issue12_Fall2019.pdf}{Jul 2019}
	\item \Presentation{Particle physics laboratory uses GitLab to connect researchers from across the globe}{GitLab.com}{https://web.archive.org/web/20181019022114/https://about.gitlab.com/customers/cern/}{Oct 2018}
	\item \Presentation{How do the LHC Experiments work? (in American Sign Language)}{CERN YouTube Channel, part of the former CERN Microcosm Exhibit}{https://youtu.be/BaGjAruqFec}{Feb 2017}
	\item \Presentation{TV: Not Quite Dead, But Time to Pull the Plug?}{Forbes}{http://www.forbes.com/sites/quora/2014/11/25/tv-not-quite-dead-but-time-to-pull-the-plug/}{Nov 2014}
	\item \Presentation{The Pros and Cons of Inbox by Gmail}{Forbes}{http://www.forbes.com/sites/quora/2014/10/24/the-pros-and-cons-of-inbox-by-gmail/}{Nov 2014}
	\item \Presentation{My Inner Voice as a Deaf Person}{Thought Catalog}{http://thoughtcatalog.com/giordon-stark/2014/09/my-inner-voice-as-a-deaf-person}{Sep 2014}
	\item \Presentation{Best of Class 2008}{Palm Beach Post (see attachments)}{}{Apr 2008}
	\item \Presentation{All-American Scholar-at-Large}{United States Achievement Academy (see attachments)}{}{Apr 2006}
\end{itemize}

\subsection{Examples showing innovation \noneyet} \label{ssec:examples-showing-innovation-noneyet}
%within e.g., education, research or other area
\subsection{Examples showing entrepreneurship \noneyet} \label{ssec:examples-showing-entrepreneurship-noneyet}
\subsection{List of patents \noneyet} \label{ssec:list-of-patents-noneyet}

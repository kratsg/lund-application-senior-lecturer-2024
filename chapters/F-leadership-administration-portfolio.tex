\chapter{Portfolio of Leadership and Administration}

\section{Summary of leadership and administration} \label{sec:summary-of-leadership-and-administration}

Having actively engaged in numerous leadership roles within the ATLAS Collaboration, my contributions have been instrumental in advancing collaborative efforts in particle physics research. Serving as the convener for the SUSY Summaries subgroup, I played a pivotal role in synthesizing key findings and facilitating knowledge dissemination within the community. Additionally, my responsibilities as the contact person for critical areas such as Monte Carlo Production and Analysis Preservation in SUSY and Exotics underscore my commitment to ensuring the integrity and accessibility of research outcomes. Moreover, my involvement with the Combinations Task Force demonstrates a dedication to enhancing the synergy of diverse analytical approaches. Furthermore, serving on the Early Career Scientist Board committee reflects my commitment to fostering the growth and development of emerging talent within the field. Through these multifaceted leadership roles, I have demonstrated a steadfast dedication to advancing international particle physics collaborations and pushing the boundaries of scientific exploration. I hope to be able to get formal training in the future to better improve my ability to lead and inspire my team and colleagues.

\section{Leadership and administration -- personal reflection} \label{sec:leadership-and-administration-personal-reflection}

In the intricate realm of international particle physics collaborations such as ATLAS, effective leadership and administration serve as the bedrock for success and advancement. These collaborations bring together diverse expertise, resources, and perspectives from across the globe to tackle some of the most profound questions about the universe. Leadership provides the vision, direction, and coordination necessary to align these disparate elements towards common goals, fostering synergy and innovation. Meanwhile, adept administration ensures the smooth operation of complex projects, managing logistics, finances, and communication channels with precision. Together, leadership and administration create an environment conducive to breakthrough discoveries, enabling scientists to push the boundaries of knowledge and deepen our understanding of the fundamental fabric of reality.

The ATLAS Collaboration consists of over 6000 members and over 3000 scientific authors across 182 institutions in 42 countries. The Collaboration is marked by well-defined organizational structure that allows the \textsl{service work} to be spread more equitably amongst its members. Note that unlike industry where management roles are allocated (often salaried) positions, management in a particle physics collaboration such as ATLAS is \textsl{paid} in terms of recognition, visibility, and invitiations for prestigious talks.

\subsection{Convenership} \label{ssec:convenership}

When I started my postdoc in 2018, I was nominated and joined the SUSY Combinations Task Force charged with trying to coordinate multiple analysis teams to harmonize their object definitions and kinematic selections, with the ulterior motive of being able to perform a stastistical combination. For the first year, I placed the groundwork to prepare the Collaboration to publish full statistical models by taking on leading roles in the development of the technical software needed: likelihood serialization and inference, open-source particle-level analysis framework, and analysis preservation efforts. The conveners of the SUSY Working Group recognized my efforts, created a new sub-group \enquote{Run 2 Summaries}, and appointed me as the first convener for the group until 2023. Because I had laid the groundwork during my role on the Combination Task Force and was very hands-on, I switched to delegate more of the work out. I understood that the hardest part of getting big-picture lofty projects complete, like a large-scale statistical combination of multiple analyses, is even harder if you don't have a skeleton in place... but once that infrastructure was there, it was a matter of appointing Analysis Contacts to be responsibile for the individual pieces of the efforts from pMSSM scans to Combinations to RPV-RPC reinterpretations. My leadership style here is to both train users to become experts and give them all of the tools they need to succeed.

\subsection{Examples from analysis activities} \label{ssec:examples-from-analysis-activities}
I joined the ATLAS SUSY Working Group in Jan 2015 at the University of Chicago. The first analysis I worked on, which became my thesis analysis, initially targeted SUSY via strong production (gluinos, squarks) through an optimized selection that I pioneered which ended up setting the strongest limits on the mass of gluinos and stop squarks, multiple times over. I developed the analysis design for Strong SUSY discovery regions, implemented the uncertainty estimations for theory systematics and for top-modeling systematics, and performed all of the hypothesis tests. Once I put in the effort into the underlying tooling to get the analysis in ship-shape, I started to delegate the work to incoming graduate students and shifted to become more of a consulting role. The reason I do this is to allow all the knowledge I've built up to be disseminated more effectively by training new experts in the field, and giving them the tools to effectively learn.

\subsection{Examples from instrumentation activities} \label{ssec:examples-from-instrumentation-activities}
Unlike analysis activities which are started and completed on shorter timescales, instrumentation activities such as upgrade work see a much broader, longer timescale. As such, it is typically harder to be in management as an early career scientist -- especially when some instrumentation projects started before I ever became part of the ATLAS Collaboration. In spite of this, I was able to position myself in leading roles within the U.S. on the instrumentation work I've done. In gFEX, from 2014-2018 -- I was recognized for my efforts to lead (\Cref{ssec:awards-and-distinctions-in-research-activity}) and help coordinate to keep the project on track. The trigger upgrade project was successfully installed and commissioned inside the ATLAS detector in October 2021. On the Inner Tracker upgrade work, I joined SCIPP, UC Santa Cruz and built up the institute as one of the leading institutions in the U.S. for electrical QC/QA of digital pixel modules with four chips reading out silicon sensors for the Inner System of the ATLAS detector. Argonne National Lab will assemble all of the modules, and ship them out to Lawrence Berkeley National Lab and UCSC for testing. I am proud of my leadership skills to position a state university to be on the same level as a national lab in contributing to this effort.

\subsection{Statement of Importance of Diversity, Equity, Inclusion in Leadership} \label{ssec:statement-of-importance-of-diversity-equity-inclusion-in-leadership}

\enquote{Giordon will not ever achieve beyond a 3rd-grade reading level,} all the teachers told my parents when I was young. Now, I have a Ph.D., but I'm not sure that says anything about my reading level. I continue to get discriminatory and stereotypical comments to this day. I was born with a severe-to-profound hearing loss, and I grew up calling myself \enquote{hearing impaired} or \enquote{hard-of-hearing}. The entirety of my childhood relied on the advocacy of the adults around me: my parents, my teachers, the principals, and the school district Board of Education. In South Florida, where I grew up, there wasn't a deaf community, as most of the folks there with hearing loss are the geriatric crowd (and this is still pretty true today!). I got by with the tools we found at the time, such as getting an Individual Education Plan (IEP), getting speech therapy to \textsl{fit in} with my oral environment, using an FM system for sound amplification, and trying to keep my class sizes small. Often, my parents would change the school if there were issues with access to my education. Society should adapt itself to be more accessible to its members, not the other way around.
\\
\\
Once I entered Caltech for my undergraduate studies, this story changed into developing my self-advocacy and not relying so much on others. I started calling myself \enquote{deaf} as I learned more about my community and became more involved in the disability rights social movement that has been slowly taking root. I didn't quite find my footing until towards the end of my 2nd year in college, when I learned about the availability of real-time steno captioning, often called CART (Communication Access Realtime Translation), and requested this more often. I also relied on a letter that I sent to my teachers ahead of time explaining my needs, such as requesting that the teacher never speak facing a wall because I can't lip-read the back of their head. Looking back today, I am shocked by how little access I got to the world-class education there. The extra labor I undertook made it that much harder to succeed in a system built around non-disabled people.

Since starting my physics career in particle physics at the University of Chicago, I have called myself \enquote{Deaf}, meaning a part of the \textsl{Deaf} community and culture. Within this microcosm of our society, I appreciate the power of labels. I take advantage of the American Sign Language (ASL) interpreting services and CART available to me. There is a great deal of experience gained from learning to interact with the university bureaucracy when it comes to arranging services and making sure my education, and later my work, remains accessible. The additional effort is on top of self-advocacy, and so I recognize the challenges that students coming to a university setting will face when they need to reach out for their access. I want to reduce the barriers students have to access their education, to make it easier than it has been for me.

In 2018, I started on the US ATLAS Diversity and Inclusion committee and served as a contact for anonymous concerns reported by the US members of the ATLAS Collaboration. My role was to help US ATLAS management make the US ATLAS Collaboration an inclusive environment to do science. For example, a meeting checklist~\PubCite{green!50!black}{white}{6} was put together for organizers to go through to help them make their meetings accessible and inclusive. This list had items for the organizing committee, such as ensuring the venue was physically accessible and providing guidelines for promotional material, registration, and the agenda/program itself.

In parallel to the committee work, I also try to make physics more accessible to a particular underrepresented population: Deaf people. Developing new signs for ASL is a rather tricky thing because I am not a linguistic expert in American Sign Language (or any sign language for that matter). Instead, I collaborate with a team of Deaf linguistic experts. My content expertise, combined with their ASL mastery, produced new draft signs to add to the existing lexicon of ASL~\PubCite{green!50!black}{white}{24}.

While Lund has an explicit focus on \enquote{widening participation}, the Department of Physics has not explicitly made any public statements that I could find to this effect. All of us are leaders and we need to combat the racial disparities in access to higher education. In particular, a key piece is to provide demographic information to understand where to focus these efforts. Our \textsl{normal} is not equitable, as we've seen recently during the COVID-19 pandemic, with disparate impacts on marginalized communities. From a rise of hate crimes against the Asian and BIPOC community, a rollback of rights and healthcare protections for LGBTQ+-identifying persons, to disabled Swedish citizens trying to make ends meet and have access to adequate healthcare in Sweden... we must do better.

Everyone at an institute is responsible for promoting a more inclusive environment and must be proactive about DEI. At Lund, I will enhance diversity and opportunity for individuals from historically underrepresented backgrounds and communities. I will actively seek resources to support students in my research lab and my departments, such as working on the strategic priorities outlined by the ShutDownSTEM community meeting. I will make all efforts to make sure my research is approachable to undergraduates and accessible for those where English (or Swedish) is not their L1 (a speaker's first language). And lastly, I will use my position and privilege to speak up when appropriate or otherwise make the space to allow the marginalized voices to be heard and understood. I will bake all of the above into a laboratory Code of Conduct that promotes an equitable and safe environment for science and socializing.

I believe my intersectionality as the only Deaf Physicist in academia provides me with insight into the spectrum of challenges faced by minorities and underrepresented populations. People realize that there are perspectives within particle physics that are pretty under-represented, and it would be beneficial if we took steps to ensure everything we do is accessible. If there is one thing to take away from this statement, know that I will always keep an open mind, educate myself, and work with you to dismantle systemic challenges that our community members, including me, continue to face.

\section{Academic leadership and administration -- list of qualifications} \label{sec:academic-leadership-and-administration-list-of-qualifications}
\subsection{Formal training in leadership and administration \noneyet} \label{ssec:formal-training-in-leadership-and-administration-noneyet}
\subsection{Leadership positions within academia} \label{ssec:leadership-positions-within-academia}

\begin{table}[h!]
	\centering
	\footnotesize
	\caption{Table highlighting leadership positions within academia}
	\begin{tabular}{r|>{\bfseries}ll}
		\centering
		2024-present & analysis contract & SUSY Electroweak: Radiative Decays             \\
		2020-2022    & subconvener       & Supersymmetry: Run-2 Summaries                 \\
		2019-2021    & contact person    & Common Dark Matter AMG-RECAST                  \\
		2018-2020    & contact person    & SUSY Combinations, SUSY Monte Carlo Production \\
	\end{tabular}
\end{table}

\subsection{Leadership positions outside academia} \label{ssec:leadership-positions-outside-academia}

\begin{table}[h!]
	\centering
	\footnotesize
	\caption{Table highlighting leadership positions outside academia}
	\begin{tabular}{r|>{\bfseries}lp{20em}}
		\centering
		2008-2011 & Health \& Safety Education Chair       & Youth Council for American Red Cross, San Gabriel Pomona Valley Chapter \\
		2007-2008 & Youth Member of the Board of Directors & American Red Cross, Greater Palm Beach Area Chapter                     \\
		2007-2008 & North County Committee Chair           & American Red Cross, Greater Palm Beach Area Chapter                     \\
		2006-2007 & Health \& Safety Education Chair       & Youth Council for American Red Cross, Greater Palm Beach Area Chapter   \\
	\end{tabular}
\end{table}

\subsection{Assignments on boards and committees} \label{ssec:assignments-on-boards-and-committees}

\begin{table}[h!]
	\centering
	\footnotesize
	\caption{Table highlighting assignments on boards and committees}
	\begin{tabular}{r|>{\bfseries}lp{20em}}
		\centering
		2023-2024 & committee member                       & Formation Task Force for Coordinating Panel for Software and Computing, Division of Particles and Fields \\
		2023-2024 & Editorial Board member                 & VBF diHiggs to four $b$-quarks                                                                           \\
		2022      & committee member                       & Search Committee for ATLAS Early Career Scientist Board                                                  \\
		2021-2022 & committee member                       & ATLAS Early Career Scientist Board                                                                       \\
		2020      & committee member                       & Search Committee for US ATLAS Education and Public Outreach Co-coordinators                              \\
		2018-2022 & committee member                       & US-ATLAS Diversity and Inclusion                                                                         \\
		2007-2008 & Youth Member of the Board of Directors & American Red Cross, Greater Palm Beach Area Chapter                                                      \\
	\end{tabular}
\end{table}

\subsection{Assignments concerning ethics, gender equality, work environment and environmental issues} \label{ssec:assignments-concerning-ethics-gender-equality-work-environment-and-environmental-issues}

\begin{table}[h!]
	\centering
	\footnotesize
	\caption{Table highlighting assignments that concern ethics, gender equality, work environment, and environmental issues.}
	\begin{tabular}{r|>{\bfseries}lp{20em}}
		\centering
		2022      & committee member                       & Search Committee for ATLAS Early Career Scientist Board                     \\
		2020      & committee member                       & Search Committee for US ATLAS Education and Public Outreach Co-coordinators \\
		2021-2022 & committee member                       & ATLAS Early Career Scientist Board                                          \\
		2018-2022 & committee member                       & US-ATLAS Diversity and Inclusion                                            \\
		2008-2011 & Health \& Safety Education Chair       & Youth Council for American Red Cross, San Gabriel Pomona Valley Chapter     \\
		2007-2008 & Youth Member of the Board of Directors & American Red Cross, Greater Palm Beach Area Chapter                         \\
		2007-2008 & North County Committee Chair           & American Red Cross, Greater Palm Beach Area Chapter                         \\
		2006-2007 & Health \& Safety Education Chair       & Youth Council for American Red Cross, Greater Palm Beach Area Chapter       \\
	\end{tabular}
\end{table}

\subsection{Management and cooperation expertise within other organisations outside the University such as scholarly or professional organisations \noneyet} \label{ssec:management-and-cooperation-expertise-within-other-organisations-outside-the-university-such-as-scholarly-or-professional-organisations-noneyet}

\chapter{Research Qualifications Portfolio}

\section{Summary of research/research profile} \label{sec:summary-of-research-research-profile}

\section{Research activity} \label{sec:research-activity}

% This personal reflection is to include an account of completed research projects, current research interests and plans for the future. The attachments that are necessary to confirm the contents of the personal reflection are to be attached respectively to the list of qualifications and the list of publications below. The reflection should not exceed 8 pages for professorships and senior lectureships and 4 pages for other teaching positions.
\subsection{Previous research activity}\label{ssec:previous-research-activity}
\subsection{Current research}\label{ssec:current-research}
\subsection{Plans for the future}\label{ssec:plans-for-the-future}

\section{Research experience and qualifications} \label{sec:research-experience-and-qualifications}

\subsection{Research environment and scholarly networks}\label{ssec:research-environment-and-scholarly-networks}

A detailed breakdown of key research projects that I am involved in, as well as their scholarly networks are summarized in~\Cref{ssec:important-research-collaborations}. I am a member of the ATLAS Collaboration, based at CERN. Within ATLAS, I have worked in:

\begin{itemize}
	\setlength{\itemsep}{0em}
	\item Physics Analysis Working Groups: Supersymmetry, Exotics, Higgs, and Standard Model
	\item Combined Performance Working Groups: Jet/MET
	\item Other Working Groups: Analysis Model Group, Statistics Committee, and Computing \& Software
	\item Upgrade Efforts: Level-1 Calorimeter Phase I, Inner Tracker Phase II
\end{itemize}

Outside of the ATLAS Collaboration, I have expressed interest and put some work towards other experiments in my free time:

\begin{itemize}
	\setlength{\itemsep}{0em}
	\item Belle-II: initially providing statistics support through \texttt{pyhf}, but I would like to explore \enquote{light} dark matter candidates
	\item IRIS-HEP: supporting the implementation of analysis facilities that will enable the work of many physicists in the future for the HL-LHC era
	\item SciKit-HEP: collaborating with many members on the open-sourced software and computing efforts to positively impact the experience of physics analyzers
	\item HEP Software Foundation: developed training materials (\Cref{chap:teaching-qualifications-portfolio}) to improve software literacy in HEP
	\item Other future facilities: muon collider and FCC; initially focused on developing the software infrastructure to reduce the barrier to entry for other physicists
\end{itemize}

\subsection{Supervision experience}\label{ssec:supervision-experience}
\subsubsection{Experience as a principal supervisor \noneyet}\label{sssec:experience-as-a-principal-supervisor-noneyet}
% name, year of degree, higher education institution, thesis title, assistant supervisor if applicable. Indicate the doctoral student’s current work/position}
\subsubsection{Experience as an assistant supervisor}\label{sssec:experience-as-an-assistant-supervisor}
% name, year of degree, higher education institution, thesis title, name of principal supervisor}

During my postdoc tenure at UC Santa Cruz and my studies at UChicago, I mentored graduate students across various analysis and hardware projects. I list undergraduate students that I mentored in~\Cref{ssec:supervision-at-the-bachelor-s-and-master-s-degree-levels}.

{
\footnotesize
\begin{tabular}{l|>{\bfseries}r|l|>{\itshape}p{20em}|l}
	\centering
	Sam Roberts       & TBD  & UC Santa Cruz & TBD                                                                                                                                                                                      & Jason Nielsen \\
	Hava Schwartz     & TBD  & UC Santa Cruz & TBD                                                                                                                                                                                      & Jason Nielsen \\
	Nathan Kang       & TBD  & UC Santa Cruz & TBD                                                                                                                                                                                      & Mike Hance    \\
	Jacob Johnson     & TBD  & UC Santa Cruz & TBD                                                                                                                                                                                      & Mike Hance    \\
	Yuzhan Zhao       & 2024 & UC Santa Cruz & Measurement of Collinear $W$-boson production with high transverse momentum jets at $\sqrt{s} = 13$ TeV using the ATLAS detector                                                         & Bruce Schumm  \\
	Emily Smith       & 2023 & UChicago      & A Global View of Jets with the ATLAS Detector: From Hardware Triggers to Precision Measurements and Beyond                                                                               & David Miller  \\
	Carolyn Gee       & 2023 & UC Santa Cruz & Search for Higgs Bosons Produced via Vector Boson Fusion and Decaying to a Pair of bb-quarks in Association with a High-Energy Photon in the ATLAS Detector at the Large Hadron Collider & Jason Nielsen \\
	Jeffrey Shahinian & 2020 & UC Santa Cruz & Soft leptons, hard problems: Searches for the electroweak production of supersymmetric particles in compressed mass spectra with the ATLAS Detector                                      & Mike Hance    \\
	Jacob Pasner      & 2019 & UC Santa Cruz & An Inclusive Search for the decay of boosted Higgs Bosons in the $H\to b\bar{b}$ channel with the ATLAS Detecto                                                                          & Jason Nielsen \\
\end{tabular}
}

\subsubsection{Experience as a supervisor of postdoctoral researchers \noneyet}\label{sssec:experience-as-a-supervisor-of-postdoctoral-researchers-noneyet}
% name, period, research field, funding}
\subsection{Participation in the organisation of scholarly symposia and conferences}\label{ssec:participation-in-the-organisation-of-scholarly-symposia-and-conferences}
{
	\footnotesize
	\begin{tabular}{r|ll}
		\centering
		\textit{Date}   & \textit{Conference}                                            & \textit{Link}                                                            \\
		\hline
		Mar 19, 2024    & ATLAS Upgrade Week - Reporting and Database Interfaces Session & \href{https://indico.cern.ch/event/1387160/}{\faIcon{external-link-alt}} \\
		Dec 4-8, 2023   & pyhf Users and Developers Workshop 2023                        & \href{https://indico.cern.ch/event/1294577/}{\faIcon{external-link-alt}} \\
		Nov 8, 2022     & Town Hall Event with Management and Physics Coordination       & \href{https://indico.cern.ch/event/1203619/}{\faIcon{external-link-alt}} \\
		Nov 10, 2021    & Town Hall Event with Management and Physics Coordination       & \href{https://indico.cern.ch/event/1086239/}{\faIcon{external-link-alt}} \\
		Sep 20-22, 2021 & SUSY Workshop 2021 (virtual)                                   & \href{https://indico.cern.ch/event/1056428/}{\faIcon{external-link-alt}} \\
		Sep 23-25, 2020 & ATLAS Exotics + SUSY Workshop 2020 (virtual)                   & \href{https://indico.cern.ch/event/898965/}{\faIcon{external-link-alt}}  \\
		Aug 6-9, 2019   & US ATLAS Physics Workshop 2019                                 & \href{https://indico.cern.ch/event/813855/}{\faIcon{external-link-alt}}  \\
	\end{tabular}
}

\subsection{Assignments as editor of a journal or other publication \noneyet}\label{ssec:assignments-as-editor-of-a-journal-or-other-publication-noneyet}
\subsection{Important research collaborations}\label{ssec:important-research-collaborations}
% state the scope of the research, key people and funding if applicable}
\subsection{Assessment of others’ work \noneyet}\label{ssec:assessment-of-others-work-noneyet}
% grading committee assignments, expert assignments, referee assignments, peer review, assignments as faculty examiner, reviewer}
\subsection{Awards and distinctions in research activity}\label{ssec:awards-and-distinctions-in-research-activity}

Below are a list of awards and distinctions associated with my research activity. Some of these awards additional awarded money to be used for research, and detailed in~\Cref{sec:research-grants}.

\Award{UC Santa Cruz Outstanding Postdoctoral Fellow Award}{May 2022}{\enquote{The two postdoctoral scholars chosen by the selection committee to receive the award will exhibit the following:
		\begin{itemize}
			\setlength{\itemsep}{0em}
			\item Excellent research and/or creative activity, showing strong evidence of research/creative innovation and productivity (e.g., scholarly distinctions, publications, presentations, inventions, exhibits, performances, products), as well as the nominee’s research/creative innovation and productivity having a significant or potentially significant impact on the field and/or society more broadly
			\item Leadership and/or strong service (e.g., preparing manuscripts, funding applications, organizing workshops or conferences, volunteering in professional and/or other organizations, etc.)
			\item Effective mentorship by advising graduate students, undergraduate students, and/or any other group in a professional setting
			\item Support and fostering of equity, diversity, and inclusion within their research group, department, institution, community, and/or field through their research/creativity, mentoring/teaching, and/or service activities
		\end{itemize}} [Citation from UC Santa Cruz]}
\Award{Springer Thesis Award}{Aug 2019}{\enquote{The series \enquote{Springer Theses} brings together a selection of the very best Ph.D. theses from around the world and across the physical sciences. Nominated and endorsed by two recognized specialists, each published volume has been selected for its scientific excellence and the high impact of its contents for the pertinent field of research. For greater accessibility to non-specialists, the published versions include an extended introduction, as well as a foreword by the student’s supervisor explaining the special relevance of the work for the field. As a whole, the series will provide a valuable resource both for newcomers to the research fields described, and for other scientists seeking detailed background information on special questions. Finally, it provides an accredited documentation of the valuable contributions made by today’s younger generation of scientists.} [Citation from Springer]}
\Award{Nathan Sugarman Award for Excellence in Graduate Student Research}{May 2017}{\enquote{For his technical contributions and creative insights in the design and prototyping of a new high-speed electronics trigger system for Lorentz-boosted massive particles for the ATLAS Experiment.} [Citation from University of Chicago]}
\Award{US ATLAS Outstanding Graduate Student Award}{Jun 2016}{\enquote{In recognition of your exceptionally broad and noteworthy contributions to the ATLAS experiment. In particular, we recognize your critical contributions to the electronics design and prototyping for a new high-speed trigger electronics system for the Phase 1 upgrade, software development, leadership in the creation of a new method to search for Supersymmetry, and software education.} [Citation from US ATLAS]}
\Award{Young Researchers’ Symposium Award for Best Poster Presentation}{Nov 2015}{This was awarded for the best poster presentation of on-going research at Brookhaven National Lab, sponsored by the American Physical Society. The poster focused on the research I was doing as part of the US DOE SCGSR fellowship described below.}
\Award{U.S. DOE, Office of Science Graduate Student Research Fellowship}{Oct 2015}{This was given for the project titled \enquote{Boosted object hardware trigger development and testing for the Phase I upgrade of the ATLAS Experiment at Brookhaven National Lab} in collaboration with BNL Omega Group leader Dr. Michael Begel. The goal of the award is to enable early career scientists to be relocated temporarily at a national lab, creating a pathway to advance Ph.D. thesis research while using state-of-the-art facilities and cutting-edge scientific instrumentation. There is the added benefit of being able to expand my professional network and develop more opportunities for my future in HEP.}
\Award{US LHC Users Association Lightning Round winner}{Nov 2014}{Early career scientists are invited to give rapid-fire presentations on their Ph.D. research. A select few who win this \enquote{lightning round} of talks are then invited to join other scientists and colleagues in Washington D.C. on a policy-advocacy trip to communicate the importance of funding HEP research. Direct communication with U.S. Congress is critical to explain the current status of HEP projects, and inform our representatives about the impact that funding our research has in the U.S. and globally.}

\section{List of Publications} \label{sec:list-of-publications}

Publications are to be listed in chronological order (most recent first) and sorted under the headings below. The DOI code for each publication is to be provided where applicable.

\subsection{Published original articles in referee-assessed international journals}\label{ssec:published-original-articles-in-referee-assessed-international-journals}
\subsection{Overview articles and other invited articles in international journals \noneyet}\label{ssec:overview-articles-and-other-invited-articles-in-international-journals-noneyet}
\subsection{Books, book chapters}\label{ssec:books-book-chapters}
\subsection{Other articles and reports published in international journals \noneyet}\label{ssec:other-articles-and-reports-published-in-international-journals-noneyet}

\subsection{Scholarly articles and reports published in Swedish \noneyet}\label{ssec:scholarly-articles-and-reports-published-in-swedish-noneyet}

\subsection{Popular science articles/presentations}\label{ssec:popular-science-articles-presentations}

\subsection{Conference papers}\label{ssec:conference-papers}

\subsection{Manuscripts (submitted manuscripts are to be listed first, followed by works in progress)}\label{ssec:manuscripts-submitted-manuscripts-are-to-be-listed-first-followed-by-works-in-progress}

\section{Research Grants} \label{sec:research-grants}

As a postdoc, I am not eligible to be a principal (or co-applicant) on research grants from national funding agencies such as Department of Energy (DOE) or National Science Foundation (NSF) directly. I have, however, received money in recognition of my research efforts which I have detailed below, as described~\Cref{sec:awards-and-distinctions}.

\begin{table}[h!]
	\footnotesize
	\centering
	\caption{Research grants for the past five years, excluding the current year (2024). All currency in USD. Total awarded amount listed as well as personal allocation.}
	\begin{tabular}{p{8em}|p{4em}|p{4em}|p{4em}|p{4em}|p{4em}|p{4em}|r}
		\rowcolor{black!30}
		\textbf{Name of project and PI (principal investigator)}  & \textbf{2018} & \textbf{2019} & \textbf{2020} & \textbf{2021} & \textbf{2022} & \textbf{2023}   & \textbf{Funder} \\
		\hline
		pyhf Users and Developers Workshop 2023 (G. Stark et al.) & 0             & 0             & 0             & 0             & 0             & 10 000 / 752.10 & NumFocus        \\
	\end{tabular}
\end{table}

\chapter{Research Qualifications Portfolio}

\section{Summary of research/research profile} \label{sec:summary-of-research-research-profile}

\section{Research activity} \label{sec:research-activity}

% This personal reflection is to include an account of completed research projects, current research interests and plans for the future. The attachments that are necessary to confirm the contents of the personal reflection are to be attached respectively to the list of qualifications and the list of publications below. The reflection should not exceed 8 pages for professorships and senior lectureships and 4 pages for other teaching positions.
\subsection{Previous research activity}\label{ssec:previous-research-activity}
\subsection{Current research}\label{ssec:current-research}
\subsection{Plans for the future}\label{ssec:plans-for-the-future}

\section{Research experience and qualifications} \label{sec:research-experience-and-qualifications}

\subsection{Research environment and scholarly networks}\label{ssec:research-environment-and-scholarly-networks}

A detailed breakdown of key research projects that I am involved in, as well as their scholarly networks are summarized in~\Cref{ssec:important-research-collaborations}. I am a member of the ATLAS Collaboration, based at CERN. Within ATLAS, I have worked in:

\begin{itemize}
	\setlength{\itemsep}{0em}
	\item Physics Analysis Working Groups: Supersymmetry, Exotics, Higgs, and Standard Model
	\item Combined Performance Working Groups: Jet/MET
	\item Other Working Groups: Analysis Model Group, Statistics Committee, and Computing \& Software
	\item Upgrade Efforts: Level-1 Calorimeter Phase I, Inner Tracker Phase II
\end{itemize}

Outside of the ATLAS Collaboration, I have expressed interest and put some work towards other experiments in my free time:

\begin{itemize}
	\setlength{\itemsep}{0em}
	\item Belle-II: initially providing statistics support through \texttt{pyhf}, but I would like to explore \enquote{light} dark matter candidates
	\item IRIS-HEP: supporting the implementation of analysis facilities that will enable the work of many physicists in the future for the HL-LHC era
	\item SciKit-HEP: collaborating with many members on the open-sourced software and computing efforts to positively impact the experience of physics analyzers
	\item HEP Software Foundation: developed training materials (\Cref{chap:teaching-qualifications-portfolio}) to improve software literacy in HEP
	\item Other future facilities: muon collider and FCC; initially focused on developing the software infrastructure to reduce the barrier to entry for other physicists
\end{itemize}

\subsection{Supervision experience}\label{ssec:supervision-experience}
\subsubsection{Experience as a principal supervisor \noneyet}\label{sssec:experience-as-a-principal-supervisor-noneyet}
% name, year of degree, higher education institution, thesis title, assistant supervisor if applicable. Indicate the doctoral student’s current work/position}
\subsubsection{Experience as an assistant supervisor}\label{sssec:experience-as-an-assistant-supervisor}
% name, year of degree, higher education institution, thesis title, name of principal supervisor}

During my postdoc tenure at UC Santa Cruz and my studies at UChicago, I mentored graduate students across various analysis and hardware projects. I list undergraduate students that I mentored in~\Cref{ssec:supervision-at-the-bachelor-s-and-master-s-degree-levels}.

{
\footnotesize
\begin{tabular}{l|>{\bfseries}r|l|>{\itshape}p{20em}|l}
	\centering
	Sam Roberts       & TBD  & UC Santa Cruz & TBD                                                                                                                                                                                      & Jason Nielsen \\
	Hava Schwartz     & TBD  & UC Santa Cruz & TBD                                                                                                                                                                                      & Jason Nielsen \\
	Nathan Kang       & TBD  & UC Santa Cruz & TBD                                                                                                                                                                                      & Mike Hance    \\
	Jacob Johnson     & TBD  & UC Santa Cruz & TBD                                                                                                                                                                                      & Mike Hance    \\
	Yuzhan Zhao       & 2024 & UC Santa Cruz & Measurement of Collinear $W$-boson production with high transverse momentum jets at $\sqrt{s} = 13$ TeV using the ATLAS detector                                                         & Bruce Schumm  \\
	Emily Smith       & 2023 & UChicago      & A Global View of Jets with the ATLAS Detector: From Hardware Triggers to Precision Measurements and Beyond                                                                               & David Miller  \\
	Carolyn Gee       & 2023 & UC Santa Cruz & Search for Higgs Bosons Produced via Vector Boson Fusion and Decaying to a Pair of bb-quarks in Association with a High-Energy Photon in the ATLAS Detector at the Large Hadron Collider & Jason Nielsen \\
	Jeffrey Shahinian & 2020 & UC Santa Cruz & Soft leptons, hard problems: Searches for the electroweak production of supersymmetric particles in compressed mass spectra with the ATLAS Detector                                      & Mike Hance    \\
	Jacob Pasner      & 2019 & UC Santa Cruz & An Inclusive Search for the decay of boosted Higgs Bosons in the $H\to b\bar{b}$ channel with the ATLAS Detecto                                                                          & Jason Nielsen \\
\end{tabular}
}

\subsubsection{Experience as a supervisor of postdoctoral researchers \noneyet}\label{sssec:experience-as-a-supervisor-of-postdoctoral-researchers-noneyet}
% name, period, research field, funding}
\subsection{Participation in the organisation of scholarly symposia and conferences}\label{ssec:participation-in-the-organisation-of-scholarly-symposia-and-conferences}
{
	\footnotesize
	\begin{tabular}{r|ll}
		\centering
		\textit{Date}   & \textit{Conference}                                            & \textit{Link}                                                            \\
		\hline
		Mar 19, 2024    & ATLAS Upgrade Week - Reporting and Database Interfaces Session & \href{https://indico.cern.ch/event/1387160/}{\faIcon{external-link-alt}} \\
		Dec 4-8, 2023   & pyhf Users and Developers Workshop 2023                        & \href{https://indico.cern.ch/event/1294577/}{\faIcon{external-link-alt}} \\
		Nov 8, 2022     & Town Hall Event with Management and Physics Coordination       & \href{https://indico.cern.ch/event/1203619/}{\faIcon{external-link-alt}} \\
		Nov 10, 2021    & Town Hall Event with Management and Physics Coordination       & \href{https://indico.cern.ch/event/1086239/}{\faIcon{external-link-alt}} \\
		Sep 20-22, 2021 & SUSY Workshop 2021 (virtual)                                   & \href{https://indico.cern.ch/event/1056428/}{\faIcon{external-link-alt}} \\
		Sep 23-25, 2020 & ATLAS Exotics + SUSY Workshop 2020 (virtual)                   & \href{https://indico.cern.ch/event/898965/}{\faIcon{external-link-alt}}  \\
		Aug 6-9, 2019   & US ATLAS Physics Workshop 2019                                 & \href{https://indico.cern.ch/event/813855/}{\faIcon{external-link-alt}}  \\
	\end{tabular}
}

\subsection{Assignments as editor of a journal or other publication \noneyet}\label{ssec:assignments-as-editor-of-a-journal-or-other-publication-noneyet}
\subsection{Important research collaborations}\label{ssec:important-research-collaborations}
In what follows below, I provide a non-exhaustive set of examples of my research collaboration networks across the world.
I am fortunate to have supportive supervisors in both my graduate studies and my postdoctoral tenure.
However, since I typically collaborate with people and not institutions and people change institutions over the course of a particular research project, a \enquote{collaborating institution} is the institute(s) that my colleagues are part of at the time I worked with them.\\

% state the scope of the research, key people and funding if applicable}
\PubCite{black}{white}{Funding:} The US Department of Energy has supported my instrumentation projects at the University of Chicago, Brookhaven National Lab, and UC Santa Cruz. The US National Science Foundation has supported my analysis and software projects at the University of Chicago and UC Santa Cruz.\\

\PubCite{red}{white}{Experimental Analyses}\\

\ProjectSummary{Summaries of ATLAS Run 2 searches for supersymmetry}{2016-present}{ATLAS Experiment}{CERN, Switzerland}{Argonne, Arlington UT, Barcelona, Beijing IHEP, Belgrade IP, Berkeley LBNL, Bern, Bologna, Bucharest IFIN-HH, CERN, Cambridge, Carleton, Chicago, DESY, Edinburgh, Freiburg, Geneva, G\"ottingen, Harvard, Hong Kong HKU, IJCLab, Lecce, Marseille CPPM, Montreal, Munich LMU, Munich MPI, NIP-UPD, NYU New York, Oklahoma, Oregon, Oslo, Oxford, Pennsylvania, RAL, Roma I, Saclay CEA, SLAC, SYSU, \textbf{Santa Cruz UC}, Stockholm, Sussex, TRIUMF, TUM, Tokyo ICEPP, Udine, Urbana UI, Valencia, Wisconsin, Witwatersrand; ATLAS SUSY Working Group}

\begin{itemize}
	\item Created the ATLAS SUSY Run 2 Summaries subgroup in 2020.
  \item Led multiple analysis teams on: pMSSM scans (General, Electroweak, and 3rd Generation); Combinations (Electroweak, 3rd Generation); $g-2$ Summaries; and RPV-RPC reinterpretation
  \item Goal of providing global summaries of the sensitivities of existing ATLAS searches for new physics
  \item Write public notes documenting necessary procedures for theorists to reinterpret and reuse published results
  \item Spearheaded the movement of publishing statistical probability models within the ATLAS Collaboration
  \item Planned and executed the first large-scale statistical combination of fourteen analyses
\end{itemize}

\ProjectSummary{Strong and Electroweak production of supersymmetric particles}{2015-present}{ATLAS Experiment}{CERN, Switzerland}{Adelaide, Barcelona, Berkeley LBNL, Berlin HU, Bologna, Bucharest IFIN-HH, CERN, CERN Tier-0, Cambridge, Carleton, \textbf{Chicago}, DESY, Duke, Edinburgh, Geneva, Hong Kong HKU, KEK, Kyoto, La Plata, Liverpool, London UC, Lund, Manchester, Marseille CPPM, Massachusetts, McGill, Michigan, Michigan SU, Montreal, Munich LMU, Nagoya, Nijmegen, Nikhef, Oklahoma SU, Oregon, Oslo, Oxford, Paris LPNHE, Pennsylvania, RAL, SLAC, \textbf{Santa Cruz UC}, Seattle Washington, Sheffield, Stockholm, Stony Brook, Sussex, TRIUMF, TUM, Tokyo ICEPP, Tokyo Tech, UCI, Urbana UI, Valencia, Vancouver UBC, Victoria; ATLAS SUSY Working Group}

\begin{itemize}
  \item Joined the ATLAS SUSY Working Group in Jan 2015 (Chicago).
  \item Initially targeted SUSY via strong production (gluinos, squarks) through pioneering analyses that set the strongest limits on the mass of gluinos and stop squarks, multiple times over
  \item I developed the analysis design for Strong SUSY discovery regions, implemented the uncertainty estimations for theory systematics and for top-modeling systematics, and performed all of the hypothesis tests
  \item Switched to electroweak production in 2018 (gauge mediated supersymmetry breaking, higgsinos, charginos, neutralinos) through analyses that targeted final states with multiple $b$-jets, and compressed scenarios
  \item I used my software and statistical expertise to contribute to the analysis design, supervised students working on these analyses, and ensured the ability to perform statistical combinations and reinterpretations needed by the Summaries efforts above
\end{itemize}

\PubCite{magenta}{white}{Instrumentation Upgrade}\\

\ProjectSummary{Phase-II Inner Tracker Upgrade: Production Database}{2018-2024}{ATLAS Experiment}{CERN, Switzerland}{KEK (JP); IFAE Barcelona, IFIC/CSIC-UV (ES); Genova, Milano (IT); TU Dortmund, MPP, DESY, Wuppertal, Bonn (DE); STFC, Glasgow, Liverpool, Sheffield, Oxford, Manchester (GB); SFU, TRIUMF, Montreal, Victoria, UBC, Carleton (CA); Lund (SE); Copenhagen (DK); Geneve, CERN, Bern (CH); CNRS (FR); Czech Academy of Sciences, Unicorn College (CZ); UPenn, ANL, LBNL, \textbf{UCSC}, Oklahoma, SLAC (US); PUJ (CO); JSI (SI); Johannesburg (ZA); INFN Genova (IT); LPNHE, IJCLab, Saclay CEA, APC, CNRS/IN2P3 (FR); Bergen, Oslo (NO); UWA (GR); ATLAS Upgrade Working Group}\\

\begin{itemize}
  \item I joined the ATLAS ITk Production Database team in 2019. The Production Database is designed to keep track of components, tests, assemblies, and shipments for all parts used to build the Inner Tracker detector.
  \item As of April 2024, there are over 9 million test results, 1 million individual components, and 10 thousand shipments.
  \item I wrote a \texttt{Python} package called \href{https://itkdb.docs.cern.ch/latest/}{\texttt{itkdb}} which handled communication with the database API for other ITk users and developers. Around 4 million requests are made to the API on a monthly basis, and \texttt{itkdb} accounts for 80\% of all traffic.
  \item I developed a read-only database backup server (full stack), complete with federated authentication and e-group access control logic, for ITk users to prepare weekly reports and keep tabs on production flow metrics and identify data integrity issues with data in the database.
  \item Using my expertise with software and databases, I worked across various groups within ITk to ramp up their usage of the Production Database and ensure that we continue to be on track for installation and commissioning by 2029 when we enter the HL-LHC era.
\end{itemize}

\ProjectSummary{Phase-II Inner Tracker Upgrade: Module QC/QA}{2014-present}{ATLAS Experiment}{CERN, Switzerland}{CERN (CH); Tokyo ICEPP, KEK (JP); LPNHE, IJCLab, Saclay CEA (FR); Seigen, Bonn, G\"{o}ttingen, Munich (DE); Bologna, Genova, Milano, Trento (IT); Liverpool, Glasgow, Oxford (GB); LBL, \textbf{UCSC}, Oklahoma, ANL (US); Bergen (NO) ; ATLAS Upgrade Working Group}\\

\begin{itemize}
  \item I joined UCSC in 2018 and completely built up our institutional infrastructure to perform quality control and electrical testing for one third of all Inner System modules being produced in the U.S.
  \item I coordinated with other institutions to develop a \enquote{local database} (LocalDB) to manage the large amounts of testing data that we generate for each module as part of the electrical testing procedure; and to make it easier for module testing sites to synchronize their data with the Production Database
  \item Developed and maintain a software suite of tooling for the QC/QA workflow: \href{https://pypi.org/project/module-qc-tools/}{\texttt{module-qc-tools}} (measurements), \href{https://pypi.org/project/module-qc-analysis-tools/}{\texttt{module-qc-analysis-tools}} (analysis), \href{https://pypi.org/project/module-qc-data-tools/}{\texttt{module-qc-data-tools}} (core data tools), \href{https://pypi.org/project/module-qc-database-tools/}{\texttt{module-qc-database-tools}} (core local database tools), \href{https://pypi.org/project/module-qc-nonelec-gui/}{\texttt{module-qc-nonelec-gui}} (non-electrical testing)
  \item Wrote the Python-C++ bindings for the hardware control library (\href{https://gitlab.cern.ch/berkeleylab/labRemote/}{\texttt{labRemote}}) used by multiple collaborating institutions
  \item Developing the Python-C++ bindings for the readout system (\href{https://yarr.web.cern.ch}{\texttt{YARR}})
  \item In addition to the above maintenance and development, I led training to teach collaborators how to manage software packages and use industry-standard best-practices to assist me with the development and maintenance efforts for the packages listed above
\end{itemize}

\PubCite{darkgray}{white}{Software \& Computing}\\

\ProjectSummary{Statistical Software: pyhf}{2018-present}{IRIS-HEP / scikit-hep}{online / GitHub}{University of Wisconsin-Madison, \textbf{UC Santa Cruz}, TUM}

\begin{itemize}
  \item In 2018, me and two of my closest colleagues (M. Feickert, L. Heinrich) came together and developed \href{https://scikit-hep.org/pyhf/}{\texttt{pyhf}}, a pure-python implementation of the \texttt{HistFactory} statistical model for multi-bin histogram-based analysis.
  \item This package is one of my proudest successes, with more than \href{https://scikit-hep.org/pyhf/citations.html}{100 citations} in the years that followed from multiple (current and future) experimental collaborations including DUNE, Belle-II, FCC, Muon Collider; as well as from theorists and phenomenologists around the world.
  \item We recently hosted the first Users and Developers workshop (\Cref{ssec:participation-in-the-organisation-of-scholarly-symposia-and-conferences}) in order to recruit more developers and ensure the longevity of our efforts
\end{itemize}

\ProjectSummary{Analysis Frameworks and Tools: xAODANaHelpers}{2014-2023}{ATLAS Experiment}{CERN, Switzerland}{Brookhaven National Laboratory (BNL), CERN, Cavendish Laboratory, University of Cambridge, Departement de Physique Nucleaire et Corpusculaire, Universite de Geneve, Department of Physics and Astronomy, University College London, Department of Physics and Astronomy, University of Pennsylvania, Department of Physics, University of Toronto, Duke University, Department of Physics, Harvard University, INFN Genova and Universita' di Genova, Dipartimento di Fisica, Jozef Stefan Institute, Laboratoire de Physique Nucleaire et de Hautes Energies (LPNHE), Sorbonne Universite, Universite de Paris, CNRS/IN2P3, Lawrence Berkeley National Laboratory, UC Berkeley, Lunds universitet, Fysiska institutionen, McGill University, Department of Physics, Nevis Laboratory, Columbia University, Royal Institute of Technology (KTH), \textbf{SCIPP, UC Santa Cruz}, SLAC National Accelerator Laboratory, Southern Methodist University, Department of Physics, TRIUMF, Universidad de Buenos Aires, Universite Paris-Saclay, CNRS/IN2P3, IJCLab, 91405, Orsay, University of Arizona, University of British Columbia, UC Irvine, \textbf{University of Chicago}, UMass Amherst, University of Melbourne, University of Oxford, University of Pittsburgh, University of Warwick, Coventry, University of Wisconsin-Madison, Yale University}

\begin{itemize}
  \item I joined ATLAS Collaboration through University of Chicago in 2014, right at the beginning of Run 2 of the LHC; and we needed an analysis framework -- fast -- so I developed and maintained \href{https://ucatlas.github.io/xAODAnaHelpers/}{\texttt{xAODAnaHelpers}}.
  \item This was recently highlighted at the \href{https://indico.cern.ch/event/1369601/contributions/5883628/}{WLCG/HSF Workshop} by the current maintainers, and xAH has been chosen as one of the EVERSE (European Virtual Institute for Research Software Excellence) pilot cases representing user analysis software in particle physics, given
    \begin{enumerate}[label=\alph*]
      \item its widespread use in a large collaboration
      \item the fact that its modular and intuitive interface fits the needs of diverse analysis use cases that require custom calibrations and objects beyond traditional physics analyses
      \item the challenges that end-user analysis software faces when relying on centrally developed tools that are updated often but still need to have full backward compatibility for ongoing analyses.
    \end{enumerate}
\end{itemize}

\subsection{Assessment of others’ work \noneyet}\label{ssec:assessment-of-others-work-noneyet}
% grading committee assignments, expert assignments, referee assignments, peer review, assignments as faculty examiner, reviewer}
\subsection{Awards and distinctions in research activity}\label{ssec:awards-and-distinctions-in-research-activity}

Below are a list of awards and distinctions associated with my research activity. Some of these awards additional awarded money to be used for research, and detailed in~\Cref{sec:research-grants}.

\Award{UC Santa Cruz Outstanding Postdoctoral Fellow Award}{May 2022}{\enquote{The two postdoctoral scholars chosen by the selection committee to receive the award will exhibit the following:
		\begin{itemize}
			\setlength{\itemsep}{0em}
			\item Excellent research and/or creative activity, showing strong evidence of research/creative innovation and productivity (e.g., scholarly distinctions, publications, presentations, inventions, exhibits, performances, products), as well as the nominee’s research/creative innovation and productivity having a significant or potentially significant impact on the field and/or society more broadly
			\item Leadership and/or strong service (e.g., preparing manuscripts, funding applications, organizing workshops or conferences, volunteering in professional and/or other organizations, etc.)
			\item Effective mentorship by advising graduate students, undergraduate students, and/or any other group in a professional setting
			\item Support and fostering of equity, diversity, and inclusion within their research group, department, institution, community, and/or field through their research/creativity, mentoring/teaching, and/or service activities
		\end{itemize}} [Citation from UC Santa Cruz]}
\Award{Springer Thesis Award}{Aug 2019}{\enquote{The series \enquote{Springer Theses} brings together a selection of the very best Ph.D. theses from around the world and across the physical sciences. Nominated and endorsed by two recognized specialists, each published volume has been selected for its scientific excellence and the high impact of its contents for the pertinent field of research. For greater accessibility to non-specialists, the published versions include an extended introduction, as well as a foreword by the student’s supervisor explaining the special relevance of the work for the field. As a whole, the series will provide a valuable resource both for newcomers to the research fields described, and for other scientists seeking detailed background information on special questions. Finally, it provides an accredited documentation of the valuable contributions made by today’s younger generation of scientists.} [Citation from Springer]}
\Award{Nathan Sugarman Award for Excellence in Graduate Student Research}{May 2017}{\enquote{For his technical contributions and creative insights in the design and prototyping of a new high-speed electronics trigger system for Lorentz-boosted massive particles for the ATLAS Experiment.} [Citation from University of Chicago]}
\Award{US ATLAS Outstanding Graduate Student Award}{Jun 2016}{\enquote{In recognition of your exceptionally broad and noteworthy contributions to the ATLAS experiment. In particular, we recognize your critical contributions to the electronics design and prototyping for a new high-speed trigger electronics system for the Phase 1 upgrade, software development, leadership in the creation of a new method to search for Supersymmetry, and software education.} [Citation from US ATLAS]}
\Award{Young Researchers’ Symposium Award for Best Poster Presentation}{Nov 2015}{This was awarded for the best poster presentation of on-going research at Brookhaven National Lab, sponsored by the American Physical Society. The poster focused on the research I was doing as part of the US DOE SCGSR fellowship described below.}
\Award{U.S. DOE, Office of Science Graduate Student Research Fellowship}{Oct 2015}{This was given for the project titled \enquote{Boosted object hardware trigger development and testing for the Phase I upgrade of the ATLAS Experiment at Brookhaven National Lab} in collaboration with BNL Omega Group leader Dr. Michael Begel. The goal of the award is to enable early career scientists to be relocated temporarily at a national lab, creating a pathway to advance Ph.D. thesis research while using state-of-the-art facilities and cutting-edge scientific instrumentation. There is the added benefit of being able to expand my professional network and develop more opportunities for my future in HEP.}
\Award{US LHC Users Association Lightning Round winner}{Nov 2014}{Early career scientists are invited to give rapid-fire presentations on their Ph.D. research. A select few who win this \enquote{lightning round} of talks are then invited to join other scientists and colleagues in Washington D.C. on a policy-advocacy trip to communicate the importance of funding HEP research. Direct communication with U.S. Congress is critical to explain the current status of HEP projects, and inform our representatives about the impact that funding our research has in the U.S. and globally.}

\section{List of Publications} \label{sec:list-of-publications}

\noindent As a member of the ATLAS Collaboration, I co-authored 819 peer-reviewed journal articles signed \textsl{ATLAS Collaboration} since 2015, and these can be viewed at \href{https://inspirehep.net/authors/1319078}{InspireHEP}.
In addition to externally-peer-reviewed publications, the collaboration publishes internally-peer-reviewed \enquote{conference notes} (CONF) in advance of conferences.
Most of these CONF notes are typically superseded by a paper of a similar title, but a list of non-superseded CONF notes can be found through the ATLAS \href{https://twiki.cern.ch/twiki/bin/view/AtlasPublic/CONFnotes}{\faIcon{external-link-alt}~Twiki}.
The publications below are presented in reverse chronological order (newest first).
The DOI code is provided where possible.
Particle physics convention lists authors alphabetically.
In the below categories, \enquote{Experiment} includes publications as part of the experimental collaboration (including analysis and instrumentation); \enquote{Theory} includes pheonomenological work; and \enquote{Accessibility} includes training.

\begin{center}
	\PubCite{darkgray}{white}{Software \& Computing}%
	~\PubCite{red}{white}{Experiment}%
	~\PubCite{magenta}{white}{Theory}%
	~\PubCite{blue}{white}{Future Facilities}%
	~\PubCite{green!50!black}{white}{Accessibility}%
\end{center}

\subsection{Published original articles in referee-assessed international journals}\label{ssec:published-original-articles-in-referee-assessed-international-journals}

\setcounter{enumi}{0}
\begin{enumerate}[label=\PubCite{darkgray}{white}{\arabic*},resume]
	\item \Publication[10.21468/SciPostPhysCodeb.27]{Reduce, reuse, reinterpret: An end-to-end pipeline for recycling particle physics results}{G. Stark et al.}{SciPost Phys. Codebases}{2306.11055}{physics.hep-ex}
\end{enumerate}

\begin{enumerate}[label=\PubCite{blue}{white}{\arabic*},resume]
	\item \Publication[10.1140/epjc/s10052-023-11889-x]{Towards a muon collider}{G. Stark et al.}{The European Physical Journal C}{2303.08533}{physics.hep-ex}
\end{enumerate}

\begin{enumerate}[label=\PubCite{red}{white}{\arabic*},resume]
	\item \Publication[10.1140/epjc/s10052-023-11543-6]{Search for supersymmetry in final states with missing transverse momentum and three or more b-jets in 139 fb$^{-1}$ of proton\textendash{}proton collisions at $\sqrt{s} = 13$~TeV with the ATLAS detector}{ATLAS Collaboration}{Eur. Phys. J. C}{2211.08028}{physics.hep-ex}
	\item \Publication[10.48550/arXiv.2209.13128]{Report of the Topical Group on Physics Beyond the Standard Model at Energy Frontier for Snowmass 2021}{G. Stark et al.}{Snowmass Whitepaper}{2209.13128}{physics.hep-ph}
\end{enumerate}

\begin{enumerate}[label=\PubCite{blue}{white}{\arabic*},resume]
	\item \Publication[10.48550/arXiv.2209.01318]{Muon Collider Forum Report}{G. Stark et al.}{Snowmass Whitepaper}{2209.01318}{physics.hep-ex}
\end{enumerate}

\begin{enumerate}[label=\PubCite{green!50!black}{white}{\arabic*},resume]
	\item \Publication[10.48550/arXiv.2203.08748]{Accessibility in High Energy Physics: Lessons from the Snowmass Process}{G. Stark et al.}{Snowmass Whitepaper}{2203.08748}{physics.physics.ed-ph}
\end{enumerate}

\begin{enumerate}[label=\PubCite{magenta}{white}{\arabic*},resume]
	\item \Publication[10.48550/arXiv.2203.10057]{Data and Analysis Preservation, Recasting, and Reinterpretation}{G. Stark et al.}{Snowmass Whitepaper}{2203.10057}{physics.hep-ph}
\end{enumerate}

\begin{enumerate}[label=\PubCite{blue}{white}{\arabic*},resume]
	\item \Publication[10.48550/arXiv.2203.07646]{Strategy for Understanding the Higgs Physics: The Cool Copper Collider}{G. Stark et al.}{Snowmass Whitepaper}{2203.07646}{physics.hep-ex}
\end{enumerate}

\begin{enumerate}[label=\PubCite{darkgray}{white}{\arabic*},resume]
	\item \PublicationPub[10.17181/CERN.R6S3.0QKV]{SimpleAnalysis: Generator-level Analysis Framework}{G. Stark et al.}{PUB}{ATL-PHYS-PUB-2022-017}
\end{enumerate}

\begin{enumerate}[label=\PubCite{blue}{white}{\arabic*},resume]
	\item \Publication[10.3389/fphy.2022.897719]{Jets and Jet Substructure at Future Colliders}{G. Stark et al.}{FrontiersIn}{2203.07462}{physics.hep-ph}
\end{enumerate}

\begin{enumerate}[label=\PubCite{darkgray}{white}{\arabic*},resume]
	\item \PublicationPub{Implementation of simplified likelihoods in HistFactory for searches for supersymmetry}{G. Stark et al.}{PUB}{ATL-PHYS-PUB-2021-038}
\end{enumerate}

\begin{enumerate}[label=\PubCite{magenta}{white}{\arabic*},resume]
	\item \Publication[10.21468/SciPostPhys.12.1.037]{Publishing statistical models: Getting the most out of particle physics experiments}{G. Stark et al.}{SciPost Phys.}{2109.04981}{physics.hep-ph}
\end{enumerate}

\begin{enumerate}[label=\PubCite{red}{white}{\arabic*},resume]
	\item \Publication[10.1140/epjc/s10052-021-09749-7]{Search for chargino--neutralino pair production in final states with three leptons and missing transverse momentum in $\sqrt{s} = 13\;\mbox{TeV}$ $pp$ collisions with the ATLAS detector}{G. Stark et al.}{EPJC}{2106.01676}{physics.hep-ex}
	\item \PublicationPub[10.1051/epjconf/202024506017]{Likelihood preservation and statistical reproduction of searches for new physics}{G. Stark et al.}{EPJ Web Conf.}{ATL-PHYS-PUB-2019-029}
\end{enumerate}

\begin{enumerate}[label=\PubCite{magenta}{white}{\arabic*},resume]
	\item \Publication[10.21468/SciPostPhys.9.2.022]{Reinterpretation of LHC Results for New Physics: Status and Recommendations after Run 2}{LHC Reinterpretation Forum}{SciPost Phys.}{2003.07868}{physics.hep-ph}
\end{enumerate}

\begin{enumerate}[label=\PubCite{red}{white}{\arabic*},resume]
	\item \PublicationPub{Reproducing searches for new physics with the ATLAS experiment through publication of full statistical likelihoods}{ATLAS Collaboration}{PUB}{ATL-PHYS-PUB-2019-029}
	\item \Publication[10.1007/JHEP06\%282018\%29107]{Search for Supersymmetry in final states with missing transverse momentum and multiple $b$-jets in proton--proton collisions at $\sqrt{s} = 13\;\mbox{TeV}$ with the ATLAS detector}{G. Stark et al.}{JHEP 06 (2018) 107}{1711.01901}{physics.hep-ex}
	\item \Publication[10.1103/PhysRevD.94.032003]{Search for pair production of gluinos decaying via stop and sbottom in events with $b$-jets and large missing transverse momentum in $pp$ collisions at $\sqrt{s} = 13\;\mbox{TeV}$ with the ATLAS detector}{G. Stark et al.}{Phys. Rev. D}{1605.09318}{physics.hep-ex}
\end{enumerate}

\subsection{Overview articles and other invited articles in international journals \noneyet}\label{ssec:overview-articles-and-other-invited-articles-in-international-journals-noneyet}
\subsection{Books, book chapters}\label{ssec:books-book-chapters}

\begin{enumerate}[label=\PubCite{red}{white}{\arabic*},resume]
	\item \PublicationThesis[10.1007/978-3-030-34548-8]{The Search for Supersymmetry in Hadronic Final States Using Boosted Object Reconstruction}{G. Stark et al.}{Springer Theses}{\href{https://cds.cern.ch/record/2317296/}{\faIcon{external-link-alt}~CERN-THESIS-2018-047}}
	\item \PublicationInt{Global Feature Extractor of the Level-1 Calorimeter Trigger: ATLAS TDAQ Phase-I Upgrade gFEX Final Design Report}{G. Stark et al.}{Final Design Report}{\href{https://cds.cern.ch/record/2233958}{\faIcon{external-link-alt}~ATL-COM-DAQ-2016-184}}
\end{enumerate}

\subsection{Other articles and reports published in international journals \noneyet}\label{ssec:other-articles-and-reports-published-in-international-journals-noneyet}

\subsection{Scholarly articles and reports published in Swedish \noneyet}\label{ssec:scholarly-articles-and-reports-published-in-swedish-noneyet}

\subsection{Popular science articles/presentations}\label{ssec:popular-science-articles-presentations}

\begin{enumerate}[label=\PubCite{red}{white}{\arabic*},resume]
  \item \Presentation{Spacing out with Particle Physics}{CSD Learns}{https://atomichands.com/events/space-camp-unlimited/}{Apr 2024}
  \item \Presentation{Analysis Preservation Bootcamp}{IRIS-HEP}{https://web.archive.org/web/20200225185628/https://iris-hep.org/2020/02/17/analysis-preservation.html}{Feb 2020}
\end{enumerate}
\begin{enumerate}[label=\PubCite{green!50!black}{white}{\arabic*},resume]
  \item \Presentation{PARTY CALL PHYSICS: when access and physics collide}{A Keynote Speaker for the International Conference of Physics Students, 2021}{https://events.iaps.info/event/9/page/5-keynote-speakers}{Aug 2021}
\end{enumerate}
\begin{enumerate}[label=\PubCite{red}{white}{\arabic*},resume]
  \item \Presentation{ATLAS releases \enquote{full orchestra} of analysis instruments}{Symmetry Magazine}{https://web.archive.org/web/20210115133429/https://www.symmetrymagazine.org/article/atlas-releases-full-orchestra-of-analysis-instruments}{Jan 2021}
  \item \Presentation{Analysis Preservation Bootcamp}{IRIS-HEP}{https://web.archive.org/web/20200225185628/https://iris-hep.org/2020/02/17/analysis-preservation.html}{Feb 2020}
\end{enumerate}
\begin{enumerate}[label=\PubCite{green!50!black}{white}{\arabic*},resume]
  \item \Presentation{My life as a particle physicist}{CERN YouTube Channel, part of the former CERN Microcosm Exhibit}{https://youtu.be/3sESUT1UO6E}{Feb 2020}
\end{enumerate}
\begin{enumerate}[label=\PubCite{red}{white}{\arabic*},resume]
  \item \Presentation{New open release streamlines interactions with theoretical physicists}{ATLAS Experiment}{https://web.archive.org/web/20201221150149/https://atlas.cern/updates/news/new-open-likelihoods}{Dec 2019}
\end{enumerate}
\begin{enumerate}[label=\PubCite{green!50!black}{white}{\arabic*},resume]
  \item \Presentation{PARTY CALL PHYSICS: when access and physics collide}{2019 Meeting of the Division of Particles \& Fields of the American Physical Society}{https://indico.cern.ch/event/782953/contributions/3454898/}{Aug 2019}
\end{enumerate}
\begin{enumerate}[label=\PubCite{darkgray}{white}{\arabic*},resume]
  \item \Presentation{Particle physics laboratory uses GitLab to connect researchers from across the globe}{GitLab.com}{https://web.archive.org/web/20181019022114/https://about.gitlab.com/customers/cern/}{Oct 2018}
\end{enumerate}
\begin{enumerate}[label=\PubCite{green!50!black}{white}{\arabic*},resume]
  \item \Presentation{How do the LHC Experiments work?}{CERN YouTube Channel}{https://youtu.be/BaGjAruqFec}{Feb 2017}
\end{enumerate}

\subsection{Conference papers}\label{ssec:conference-papers}

\textbf{Conference Notes}

\begin{enumerate}[label=\PubCite{red}{white}{\arabic*},resume]
	\item \PublicationConf{ATLAS Run 2 searches for electroweak production of supersymmetric particles interpreted within the pMSSM}{ATLAS Collaboration}{CONF}{ATLAS-CONF-2023-055}
	\item \PublicationConf{Search for pair production of higgsinos in final states with at least three $b$-tagged jets using the ATLAS detector in $\sqrt{s} = 13\;\mbox{TeV}$ $pp$ collisions}{G. Stark et al.}{CONF}{ATLAS-CONF-2017-081}
	\item \PublicationConf{Search for production of supersymmetric particles in final states with missing transverse momentum and multiple $b$-jets at $\sqrt{s} = 13\;\mbox{TeV}$ proton-proton collisions with the ATLAS detector}{G. Stark et al.}{CONF}{ATLAS-CONF-2017-021}
	\item \PublicationConf{Search for pair production of gluinos decaying via top or bottom squarks in events with $b$-jets and large missing transverse momentum in $pp$ collisions at $\sqrt{s} = 13\;\mbox{TeV}$ with the ATLAS detector}{G. Stark et al.}{CONF}{ATLAS-CONF-2016-052}
	\item \PublicationConf{Search for pair-production of gluinos decaying via stop and sbottom in events with $b$-jets and large missing transverse momentum in $\sqrt{s} = 13\;\mbox{TeV}$ $pp$ collisions with the ATLAS detector}{G. Stark et al.}{CONF}{ATLAS-CONF-2015-067}
	\item \PublicationPub{Expected Performance of Boosted Higgs ($\rightarrow b\bar{b}$) Boson Identification with the ATLAS Detector at $\sqrt{s} = 13\;\mbox{TeV}$}{G. Stark et al.}{CONF}{ATL-PHYS-PUB-2015-035}
\end{enumerate}

\textbf{Proceedings}

\begin{enumerate}[label=\PubCite{darkgray}{white}{\arabic*},resume]
	\item \PublicationCode[10.21468/SciPostPhysCodeb.27-r0.5]{Codebase release 0.5 for mapyde}{G. Stark et al.}{SciPost Phys. Codebases}{\href{https://github.com/scipp-atlas/mapyde}{\faIcon{github}~scipp-atlas/mapyde}}
	\item \Publication[10.22323/1.414.0245]{pyhf: a pure-Python implementation of HistFactory with tensors and automatic differentiation}{G. Stark et al.}{PoS ICHEP2022 245}{2211.15838}{physics.hep-ph}
	\item \PublicationProc[10.1051/epjconf/202125102070]{Distributed statistical inference with pyhf enabled through funcX}{G. Stark et al.}{EPJ Web of Conferences 251, 02070 (2021)}{EPJ}
\end{enumerate}

\begin{enumerate}[label=\PubCite{green!50!black}{white}{\arabic*},resume]
	\item \Publication[10.1007/s41781-021-00069-9]{Software Training in HEP}{G. Stark et al.}{CSBS Springer}{2103.00659}{physics.ed-ph}
\end{enumerate}

\begin{enumerate}[label=\PubCite{darkgray}{white}{\arabic*},resume]
	\item \PublicationCode[10.21105/joss.02823]{pyhf: pure-Python implementation of HistFactory statistical models}{G. Stark et al.}{Journal of Open Source Software}{\href{https://github.com/scikit-hep/pyhf}{\faIcon{github}~scikit-hep/pyhf}}
	\item \PublicationCode[10.5281/zenodo.3334365]{diana-hep/pyhf: v0.1.2}{G. Stark et al.}{Zenodo}{\href{https://github.com/scikit-hep/pyhf}{\faIcon{github}~scikit-hep/pyhf}}
	\item \PublicationCode[10.21105/joss.00307]{root{\_}numpy: The interface between {ROOT} and {NumPy}}{G. Stark et al.}{The Journal of Open Source Software}{\href{https://github.com/scikit-hep/root_numpy}{\faIcon{github}~scikit-hep/root\_numpy}}
	\item \PublicationCode[10.5281/zenodo.3743307]{xAODAnaHelpers, v1.0.0}{G. Stark et al.}{Zenodo}{\href{https://github.com/UCATLAS/xAODAnaHelpers/}{\faIcon{github}~UCATLAS/xAODAnaHelpers}}
\end{enumerate}

\begin{enumerate}[label=\PubCite{red}{white}{\arabic*},resume]
	\item \PublicationProc[10.1109/NSSMIC.2015.7581865]{gFEX, the ATLAS Calorimeter Level-1 Real Time Processor}{ATLAS Collaboration}{2015 IEEE Nuclear Science Symposium and Medical Imaging Conference}{ATL-DAQ-PROC-2015-059}
\end{enumerate}

\textbf{Conference Talks on behalf of the ATLAS Collaboration in national and international meetings}

\begin{enumerate}[label=\PubCite{red}{white}{\arabic*},resume]
  \item \Presentation{Searches for Supersymmetry with the ATLAS Detector}{Parallel session; 30th International Symposium on Lepton Photon Interactions at High Energies}{https://indico.cern.ch/event/949705/contributions/4556056/}{Jan 2022}
  \item \Presentation{SUSY in ATLAS Experiment}{Plenary talk; XXVIIth International Conference on Supersymmetry and Unification of Fundamental Interactions (SUSY 2019)}{https://cds.cern.ch/record/2675305}{May 2019}
  \item \Presentation{Searches in High Pile-up Environment}{ATLAS P \& P Week, Topical Physics Plenary: Physics with high pile-up}{https://indico.cern.ch/event/698458/}{Feb 2018}
  \item \Presentation{Forward jet shapes in high pile-up}{Hadronic Final State Forum 2017}{https://indico.cern.ch/event/666007/}{Dec 2017}
  \item \Presentation{Search for production of supersymmetric particles in final states with missing transverse momentum and multiple b-jets at s=$\sqrt{13}$~TeV proton-proton collisions with the ATLAS experiment}{US LHC Users Association Annual Meeting, 2017}{https://indico.fnal.gov/event/15068/}{Nov 2017}
  \item \Presentation{Release 21 and 2017 data scrutiny — final report}{ATLAS Week (Bratislava)}{https://indico.cern.ch/event/558932}{Oct 2017}
  \item \Presentation{L1Calo Simulation Effort for Run-2 Upgrades}{Hadronic Calibration Workshop 2017}{https://indico.cern.ch/event/642438}{Sep 2017}
  \item \Presentation{Search for production of supersymmetric particles in final states with missing transverse momentum and multiple b-jets at s=$\sqrt{13}$~TeV proton-proton collisions with the ATLAS experiment}{APS Division of Particles and Fields 2017}{https://indico.fnal.gov/event/11999/session/10/contribution/191}{Aug 2017}
  \item \Presentation{The Calorimeter Global Feature Extractor (gFEX) for the Phase-I Upgrade of the ATLAS experiment}{APS Division of Particles and Fields 2017}{https://indico.fnal.gov/event/11999/session/21/contribution/192}{Aug 2017}
  \item \Presentation{New techniques in Boosted Object Reconstruction}{US ATLAS Workshop 2017}{https://indico.cern.ch/event/631514/contributions/2670476/}{Jul 2017}
  \item \Presentation{The Calorimeter Global Feature Extractor (gFEX) for the Phase-I Upgrade of the ATLAS experiment}{US ATLAS Workshop 2017}{https://indico.cern.ch/event/631514/contributions/2656535/}{Jul 2017}
  \item \Presentation{Search for production of supersymmetric particles in final states with missing transverse momentum and multiple b-jets at s=$\sqrt{13}$~TeV proton-proton collisions with the ATLAS experiment}{US ATLAS Workshop 2017}{https://indico.cern.ch/event/631514/contributions/2656482/}{Jul 2017}
  \item \Presentation{gFEX jet calibration}{Hadronic Final State Forum 2016}{https://indico.cern.ch/event/565930/}{Dec 2016}
  \item \Presentation{SUSY using boosted techniques at ATLAS}{BOOST 2016}{https://indico.cern.ch/event/439039}{Aug 2016}
  \item \Presentation{gFEX Online Object Kinematics and Preliminary Rates}{Hadronic Calibration Workshop 2016}{https://indico.cern.ch/event/506093/}{Sep 2016}
  \item \Presentation{Jets: An Origin Story about Origin-Correction}{Hadronic Final State Forum 2015}{https://indico.cern.ch/event/437024/}{Dec 2015}
  \item \Presentation{Multi-b-jet top-tagging optimizations in boosted-0L region}{Hadronic Calibration Workshop 2015}{https://indico.cern.ch/event/368856/contributions/1785727/}{Sep 2015}
  \item \Presentation{gTowers as a proxy for subjets}{Hadronic Calibration Workshop 2015}{https://indico.cern.ch/event/368856/contributions/1785772/}{Sep 2015}
  \item \Presentation{Jet Reclustering in a Single Bound}{Hadronic Calibration Workshop 2015}{https://indico.cern.ch/event/368856/contributions/1785890/}{Sep 2015}
  \item \Presentation{JetSubstructureUtils: Substructure Taggers, Subjet Finding, and Shower Deconstruction}{Hadronic Calibration Workshop 2015}{https://indico.cern.ch/event/368856/contributions/1785785/}{Sep 2015}
  \item \Presentation{Boosted object hardware trigger development and testing for the Phase I upgrade of the ATLAS Experiment}{US ATLAS Physics Workshop}{https://indico.cern.ch/event/388328}{Jun 2015}
  \item \Presentation{Boosted object hardware trigger development and testing for the Phase I upgrade of the ATLAS Experiment}{APS April Meeting}{http://meetings.aps.org/Meeting/APR15/Session/C16.6}{Apr 2015}
  \item \Presentation{The Global Feature Extraction (gFEX) module and its Physics Impact}{US LHC Users Association Annual Meeting 2014}{https://indico.hep.anl.gov/indico/conferenceDisplay.py?confId=410}{Nov 2014}
\end{enumerate}

\textbf{Colloquia, seminars, and invited talks}

\begin{itemize}[label=\strut]
	\setlength{\itemsep}{0em}
  \item\PubCite{blue!75!white}{white}{Colloquium} \InvitedTalk{CEA Paris-Saclay}{March 2024}
  \item\PubCite{blue!75!white}{white}{Colloquium} \InvitedTalk{University of Arizona}{January 2024}
  \item\PubCite{gray!75!white}{white}{Seminar} \InvitedTalk{University of Arizona}{January 2024}
  \item\PubCite{gray!75!white}{white}{Seminar} \InvitedTalk{University of Notre Dame}{January 2024}
  \item\PubCite{gray!75!white}{white}{Seminar} \InvitedTalk{Karlsruhe Institute of Technology}{December 2023}
  \item\PubCite{blue!75!white}{white}{Colloquium} \InvitedTalk{University of Victoria}{November 2023}
  \item\PubCite{blue!75!white}{white}{Colloquium} \InvitedTalk{Wayne State University}{April 2023}
  \item\PubCite{gray!75!white}{white}{Seminar} \InvitedTalk{Southern Methodist University}{March 2023}
  \item\PubCite{gray!75!white}{white}{Seminar} \InvitedTalk{Stony Brook University}{February 2023}
  \item\PubCite{gray!75!white}{white}{Seminar} \InvitedTalk{Lund University, Sweden}{September 2022}
  \item\PubCite{gray!75!white}{white}{Seminar} \InvitedTalk{University of Washington, Bothell}{July 2022}
  \item\PubCite{gray!75!white}{white}{Seminar} \InvitedTalk{University of Pennsylvania}{February 2022}
  \item\PubCite{gray!75!white}{white}{Seminar} \InvitedTalk{The University of Cambridge, UK}{January 2022}
  \item\PubCite{gray!75!white}{white}{Seminar} \InvitedTalk{The University of Tennessee, Knoxville}{October 2021}
  \item\PubCite{gray!75!white}{white}{Seminar} \InvitedTalk{ICPS 2021}{August 2021}
  \item\PubCite{green!75!blue}{white}{Plenary} \InvitedTalk{SUSY2019}{May 2019}
  \item\PubCite{gray!75!white}{white}{Seminar} \InvitedTalk{The University of Chicago}{November 2018}
  \item\PubCite{gray!75!white}{white}{Seminar} \InvitedTalk{Columbia College Chicago}{November 2018}
\end{itemize}

\textbf{Posters}

\begin{enumerate}[label=\PubCite{red}{white}{\arabic*},resume]
  \item \Presentation{Boosted object hardware trigger development and testing for the upgrades of the ATLAS Experiment}{Poster Session ATLAS Week NYU}{https://indico.cern.ch/event/544717/}{Jul 2016}
  \item \Presentation{Search for gluinos in Run 2 using final states with boosted top quarks, b-jets and large missing energy with the ATLAS Detector}{Poster Session ATLAS Week NYU}{https://indico.cern.ch/event/544717/}{Jul 2016}
  \item \Presentation{Boosted object hardware trigger development and testing for the Phase I upgrade of the ATLAS Experiment}{Young Researchers' Symposium at BNL}{https://www.bnl.gov/bnlyrs2015/index.php}{Nov 2015}
\end{enumerate}

\subsection{Manuscripts (submitted manuscripts are to be listed first, followed by works in progress)}\label{ssec:manuscripts-submitted-manuscripts-are-to-be-listed-first-followed-by-works-in-progress}

\begin{enumerate}[label=\PubCite{blue}{white}{\arabic*},resume]
	\item \Publication{Analysis Facilities White Paper}{G. Stark et al.}{submitted to SciPost Phys.}{2404.02100}{physics.hep-ex}
\end{enumerate}
\begin{enumerate}[label=\PubCite{red}{white}{\arabic*},resume]
	\item \Publication[10.48550/arXiv.2402.08347]{A statistical combination of ATLAS Run 2 searches for charginos and neutralinos at the LHC}{ATLAS Collaboration}{submitted to PRL}{2402.08347}{physics.hep-ex}
	\item \Publication[10.1007/JHEP05(2024)106]{ATLAS Run 2 searches for electroweak production of supersymmetric particles interpreted within the pMSSM}{ATLAS Collaboration}{submitted to JHEP}{2402.01392}{physics.hep-ex}
	\item \Publication[10.48550/arXiv.2401.14922]{Search for pair production of higgsinos in events with two Higgs bosons and missing transverse momentum in $\sqrt{s}=13$ TeV $pp$ collisions at the ATLAS experiment}{ATLAS Collaboration}{submitted to PRD}{2401.14922}{physics.hep-ex}
  \item \PublicationOngoing{Collinear W+jets at 13 TeV}{ATLAS Collaboration}{internal:ANA-STDM-2020-30}{\textit{in preparation, 13 TeV version of \doi{10.1016/j.physletb.2016.12.005}}}
  \item \PublicationOngoing{SUSY grand pMSSM scan (pMSSM-19)}{ATLAS Collaboration}{internal:ANA-SUSY-2020-14}{\textit{in preparation}}
  \item \PublicationOngoing{SUSY Compressed Electroweak: Sleptons}{ATLAS Collaboration}{internal:ANA-SUSY-2023-05}{\textit{in preparation}}
  \item \PublicationOngoing{SUSY Compressed Electroweak: VBF}{ATLAS Collaboration}{internal:ANA-SUSY-2023-26}{\textit{in preparation}}
  \item \PublicationOngoing{SUSY Compressed Electroweak: Radiative Decays}{ATLAS Collaboration}{internal:ANA-SUSY-2024-12}{\textit{analysis ramping up}}
\end{enumerate}

\section{Research Grants} \label{sec:research-grants}

As a postdoc, I am not eligible to be a principal (or co-applicant) on research grants from national funding agencies such as Department of Energy (DOE) or National Science Foundation (NSF) directly. I have, however, received money in recognition of my research efforts which I have detailed in~\Cref{tab:grants}, as described~\Cref{sec:awards-and-distinctions}.

\begin{table}[h!]
	\footnotesize
	\centering
  \caption{Research grants for the past five years, excluding the current year (2024). All currency in USD. Total awarded amount listed as well as personal allocation.\label{tab:grants}}
	\begin{tabular}{p{8em}|p{4em}|p{4em}|p{4em}|p{4em}|p{4em}|p{4em}|r}
		\rowcolor{black!30}
		\textbf{Name of project and PI (principal investigator)}  & \textbf{2018} & \textbf{2019} & \textbf{2020} & \textbf{2021} & \textbf{2022} & \textbf{2023}   & \textbf{Funder} \\
		\hline
		pyhf Users and Developers Workshop 2023 (G. Stark et al.) & 0             & 0             & 0             & 0             & 0             & 10 000 / 752.10 & NumFocus        \\
	\end{tabular}
\end{table}
